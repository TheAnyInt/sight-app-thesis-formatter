%%
%% Copyright: Andy GAO (华工自动化)
%%
%%==========================================================
%%            华南理工大学博士学位论文
%%      Author: Andy GAO        Date: 2022.01.
%%==========================================================
%%
\documentclass[unicode]{SCUTthesis}
%\documentclass[draft,unicode]{SCUTthesis} %不显示图表和超链接,快速编译以核对版式
\usepackage{fontspec}       % 字体选择宏包
\usepackage{setspace}       % 间距宏包
\usepackage{ulem}           % 下划线宏包
\usepackage{CJKulem}        % 中文自动换行下划线宏包
\usepackage{fix-cm}         % Permit Computer Modern fonts at arbitrary sizes
%\usepackage{lmodern}        % 允许任意大小的字体【会改变新罗马字体的显示格式,应弃用】
\usepackage{comment}        % 分块注释宏包
\usepackage{xcolor}         % 颜色宏包 {xcolor} 包含{color}
\usepackage{array}          % 数组/向量宏包
\usepackage{longtable}      % 可分页长表格宏包
\usepackage{multirow}       % 表格中多行合并宏包
\usepackage{booktabs}       % 表格线宏包
\usepackage{graphicx}       % 图片版式宏包
\usepackage{subfig}         % 子浮动体宏包(For Compiling \subfloat)
\usepackage{amsmath,dsfont,amssymb}% 数学公式宏包, 数学符号宏包(For Compiling \triangleq)
\usepackage{mathrsfs}       % 数学花体字母宏包
\usepackage{bm}             % 公式加粗宏包
\usepackage{amstext}        % 公式中的文字编辑宏包(For Compiling \text)
\usepackage{pifont}         % 圆圈数字宏包
\usepackage{algorithmic}    % 算法宏包
%\usepackage{ccmap}          % 解决PDF复制乱码问题【只适用于PDFLaTeX编译】
\usepackage[colorlinks=true]{hyperref}
\usepackage{bookmark}       % 书签宏包
\usepackage[UTF8]{ctex}


%% 【超链接设置】
\hypersetup{
 linkcolor=blue, citecolor=blue, urlcolor=blue,
 %linkcolor=black, citecolor=black, urlcolor=black,
 anchorcolor=blue, filecolor=magenta, menucolor=red,
 unicode=true, breaklinks=false,
 bookmarksnumbered=true, bookmarksopen=true,
 pdfborder={0 0 1}, pdfstartview=FitH,
 pdftitle={论文题目},
 pdfauthor={高勇},
 pdfsubject={华南理工大学博士学位论文},
}


\makeatletter % 【定义和修改含有@符号的核心命令】
%% 【自定义指定长度的下划线】
\newcommand\dlmu[2][150mm]{\hskip1pt\underline{\hb@xt@ #1{\hss#2\hss}}\hskip3pt}
%% 【参考文献序号对齐方式】
\renewcommand{\@biblabel}[1]{[#1]\hfill}  % 参考文献序号左对齐
%\renewcommand{\@biblabel}[1]{\hfill[#1]}  % 参考文献序号右对齐(默认格式)
%% 【Because html converters don't know tabularnewline】
\providecommand{\tabularnewline}{\\}
\makeatother


%% 【定义单盲/双盲评审格式】
\newif\ifSingleBlindReview
\SingleBlindReviewtrue % 单盲评审格式【即最终上交格式】
%\SingleBlindReviewfalse % 双盲评审格式


%% 【设置正文引用子图序号为 1(a) 型,需调用宏包{subfig},正文格式为 {\ref{}\subref{}} 或{\subref*{}}】
%\DeclareSubrefFormat{parens}{#1(#2)}


%% 【设置PDF图表的兼容版本】
%\pdfoptionpdfminorversion=7


%% 【选择编译的章节】
%\includeonly{{TITLEPAGE/TitlePage},{CHAPTER3/Chapter3},{REFERENCES/References}}



\begin{document}

\institute{华南理工大学}
\degree{博士}
\submitdate{{2022}~年~{6}~月~{28}~日}
%\submitdate{} % 为空时输出编译当天日期



%================================================================================
%% 封面, 内封, 题名页, 原创性声明
%================================================================================
%%
%% Completed: 2022/01/14
%% Copyright: Andy GAO (华工自动化)
%%
%%==========================================================
%%          华南理工大学博士学位论文——扉页
%%      Author: Andy GAO        Date: 2022.01.
%%==========================================================
%%
\clearpage
\pagestyle{empty}


%%==========================================================
%% 封面
%%==========================================================
\begin{center}
%
\noindent

\vspace{2.5cm}

%
%%%%%%%%%%%%%%%%%%%%%%%%%%%%%%%%%%%%%%%%%%%%%%%%%%%%% Figure
\begin{figure}[!h]
  \centering
  \includegraphics[width=12.1cm,height=2.76cm]{TITLEPAGE/FIGs/SCUTemblem} % 校徽
  %\caption{}
  %\label{fig:}
\end{figure}
%%%%%%%%%%%%%%%%%%%%%%%%%%%%%%%%%%%%%%%%%%%%%%%%%%%%% End Figure
%

\vspace{0.5em}
{\heiti\chuhao
\makebox[9.1cm][s]{\thesissubject}
}\\

    \noindent\rule{\textwidth}{0.4pt}

    \vspace{1.5cm}

    % 论文题目
    {\heiti\erhao
       \uline{\hfill{LaTeX模板介绍与案例说明}\hfill}
    }

    \vspace{3cm}
\end{center}


\ifSingleBlindReview
%%----------------------------------------------------------
%% 单盲评审/终版提交封面
%%----------------------------------------------------------

\begin{minipage}[!h]{0.75\linewidth}
\centering
\heiti\sanhao
\makebox[4cm][s]{作者姓名}~~\uline{\hfill{高{\quad}勇}\hfill}\hfill\\
\makebox[4cm][s]{学科专业}~~\uline{\hfill{控制科学与工程}\hfill}\hfill\\
\makebox[4cm][s]{指导教师}~~\uline{\hfill{魏~武{\quad}教~授}\hfill}\hfill\\
\makebox[4cm][s]{所在学院}~~\uline{\hfill{自动化科学与工程学院}\hfill}\hfill\\
\makebox[4cm][s]{论文提交日期}~~\uline{\hfill{\thesissubmitdate}\hfill}\hfill\\
\end{minipage}



\else
%%----------------------------------------------------------
%% 双盲评审封面
%%----------------------------------------------------------

\begin{minipage}[!h]{0.75\linewidth}
\centering
\heiti\sanhao
\makebox[4cm][s]{学科专业}~~\uline{\hfill{控制科学与工程}\hfill}\\
\makebox[4cm][s]{所在学院}~~\uline{\hfill{自动化科学与工程学院}\hfill}\\
\makebox[4cm][s]{论文提交日期}~~\uline{\hfill{\thesissubmitdate}\hfill}\\
\end{minipage}

\vfill

% 论文评审结果处理办法
\newpage
\null
\noindent
\centering

\vspace{3cm}

%
%%%%%%%%%%%%%%%%%%%%%%%%%%%%%%%%%%%%%%%%%%%%%%%%%%%%% Figure
\begin{figure}[!h]
  \centering
  \includegraphics[width=\linewidth]{TITLEPAGE/FIGs/PingShenChuLi} % 评审结果处理办法
  %\caption{}
  %\label{fig:}
\end{figure}
%%%%%%%%%%%%%%%%%%%%%%%%%%%%%%%%%%%%%%%%%%%%%%%%%%%%% End Figure
%


\fi
%
\end{center}

\vfill





\ifSingleBlindReview
%%----------------------------------------------------------
%% 单盲评审/终版提交内封、题名页、授权书
%%----------------------------------------------------------
%%==========================================================
%% 英文内封
%%==========================================================
\newpage
\begin{center}
%
\noindent

\vspace{4cm}

%% 论文的英文标题
\textbf{{\xiaoerhao
Title Title Title Title Title Title Title Title Title Title Title Title Title Title Title Title
}}

\vspace{2cm}

{\sihao
A Dissertation Submitted for the Degree of Doctor of Philosophy
}

\vspace{3cm}

\begin{minipage}[!h]{0.4\linewidth}
\raggedright        % 左对齐
\textbf{\xiaosanhao
\makebox[2.8cm][s]{Candidate:}  Gao~Yong \hfill\\
\makebox[2.8cm][s]{Supervisor:}  Prof. Wei~Wu \hfill\\
}
\end{minipage}

\vspace{3.5cm}

{\xiaosanhao
South China University of Technology \par
Guangzhou, China
}

%
\end{center}

\vfill





%%==========================================================
%% 题名页
%%==========================================================
\newpage
\null

\begin{flushleft}   % 左对齐
%
\noindent
\textbf{{\heiti\sihao
分类号:TP242 {\hfill}
学校代号:10561
{\\}
学{\quad}号:20**********
}}
%
\end{flushleft}

\vspace{2.5cm}

\begin{center}      % 居中对齐
%
{\heiti\xiaoerhao
华南理工大学{\thesissubject}
}

\vspace{2cm}

%% (论文题名和副题名)
\textbf{{\heiti\xiaoyihao
华南理工大学博/硕士学位论文
{\\}{\vspace{0.5em}}
LaTeX模板介绍与案例说明
}}
%
\end{center}

\vspace{3.5cm}

\begin{minipage}[!h]{14cm}
\raggedright        % 左对齐
\songti\wuhao
\makebox[7cm][s]{作者姓名:高{\quad}勇 {\hfill}}%
\makebox[7cm][s]{指导教师姓名、职称:魏~武~~教授 {\hfill}} \\
\makebox[7cm][s]{申请学位级别:工学博士 {\hfill}}%
\makebox[7cm][s]{学科专业名称:控制科学与工程 {\hfill}} \\
\makebox[14cm][s]{研究方向:仿生机器人运动规划与智能控制 {\hfill}} \\
\makebox[7cm][s]{论文提交日期:{{\;2022\;}年{\;6\;}月{\;28\;}日}{\hfill}}%
\makebox[7cm][s]{论文答辩日期:{{\;2022\;}年{~~*~~}月{~~*~~}日}{\hfill}} \\
\makebox[7cm][s]{学位授予单位:华南理工大学 {\hfill}}%
\makebox[7cm][s]{学位授予日期:{{\;\qquad\;}年{~~\enskip~~}月{~~\enskip~~}日}{\hfill}} \\
\makebox[14cm][s]{答辩委员会成员:{\hfill}} \\
\makebox[4cm][s]{主席:\uline{\qquad 孙** \hfill}} \\
\makebox[13cm][s]{委员:\uline{\qquad 刘**、马**、魏*、刘** \hfill}} \\
\end{minipage}

\vfill





%%==========================================================
%% 原创性声明
%%==========================================================
\newpage
\null

\begin{center}
\noindent
\textbf{{\songti\erhao
华南理工大学
{\\}
学位论文原创性声明
}}
\end{center}

{%
\raggedright        % 左对齐
\parindent=1cm      % 首行缩进
\songti\sihao
\baselineskip=20pt  % 调整行距
本人郑重声明:所呈交的论文是本人在导师的指导下独立进行研究所取得的研究成果。除了文中特别加以标注引用的内容外,本论文不包含任何其他个人或集体已经发表或撰写的成果作品。对本文的研究做出重要贡献的个人和集体,均已在文中以明确方式标明。本人完全意识到本声明的法律后果由本人承担。

\bigskip
作者签名:\makebox[5cm][s]{\hfill}
日期:{{\qquad\quad}年{\qquad}月{\qquad}日}
}%



%%==========================================================
%% 使用授权书
%%==========================================================
\bigskip
\begin{center}
\noindent
\textbf{{\songti\erhao
学位论文版权使用授权书
}}
\end{center}

{%
\raggedright        % 左对齐
\parindent=1cm      % 首行缩进
\songti\sihao
\baselineskip=20pt  % 调整行距
本学位论文作者完全了解学校有关保留、使用学位论文的规定,即:研究生在校攻读学位期间论文工作的知识产权单位属华南理工大学。学校有权保存并向国家有关部门或机构送交论文的复印件和电子版,允许学位论文被查阅(除在保密期内的保密论文外);学校可以公布学位论文的全部或部分内容,可以允许采用影印、缩印或其它复制手段保存、汇编学位论文。本人电子文档的内容和纸质论文的内容相一致。

本学位论文属于:

$\Box$保密(校保密委员会审定为涉密学位论文时间:{\underline{\;~\quad}年\underline{\quad}月\underline{\quad}日}),于{\underline{\qquad} 年\underline{\quad}月\underline{\quad}日}解密后适用本授权书。

$\Box$不保密,同意在校园网上发布,供校内师生和与学校有共享协议的单位浏览;同意将本人学位论文编入有关数据库进行检索,传播学位论文的全部或部分内容。

{\quad}(请在以上相应方框内打“$\surd$”)

\bigskip
作者签名:{\qquad}\makebox[5cm][s]{\hfill}
日期:{\hfill}\par
指导教师签名:\makebox[5cm][s]{\hfill}
日期:{\hfill}\par
作者联系电话:\makebox[5cm][s]{\hfill}
电子邮箱:{\hfill}\par
联系地址(含邮编):
}%


\vfill


\fi



\baselineskip=1\baselineskip    % 调整行距
\normalsize                     % 后文字体:小四宋体


%%
%% Copyright: Andy GAO (华工自动化)
%%
%%==========================================================
%%          华南理工大学博士学位论文——扉页
%%      Author: Andy GAO        Date: 2022.01.
%%==========================================================
%%

% 把“封面”加入书签,而不用加入目录。依赖命令\usepackage{bookmark}与\usepackage[bookmarksopen=true]{hyperref}
\bookmark[level=0, named=FirstPage]{封面}



%================================================================================
%% 中英文摘要,目录,图表清单
%================================================================================
\frontmatter

%% 摘要
%%
%% Copyright: Andy GAO (华工自动化)
%%
%%==========================================================
%%          华南理工大学博士学位论文——摘要
%%       Author: Andy GAO        Date: 2022.01.
%%==========================================================
%%

%================================================================================
%% 中文摘要
%================================================================================
\begin{abstractCN}
%%
本文主要研究内容与创新成果如下。


(1)
该模板的测试环境为:操作系统Win10x64,编辑器版本为WinEdt7.0,MiKTeX版本为TeX2.9.3759,编译方式为XeLaTex。

(2)
后缀为.tex的文件打开方式选择UTF-8。

(3)
采用includeonly指令可以自由选择想要编译的某些章节,从而实现快速查看并核对这些章节的内容与版式是否达标。

(4)
XeLaTex Compilation Report (Pages: 33) {\qquad}
Errors: 0  {\qquad}  Warnings: 2  {\qquad}  Bad Boxes: 0

%%
\end{abstractCN}



%================================================================================
%% 中文关键词
%================================================================================
\keywordsCN{%
XeLaTex;
WinEdt7.0;
MiKTeX-TeX2.9.3759;
Win10x64;
关键词5
}%



%================================================================================
%% 英文摘要
%================================================================================
\begin{abstractEN}
%%
Technology energizes life, and intelligence leads the future.


(1)


(2)


(3)


(4)

%%
\end{abstractEN}



%================================================================================
%% 英文关键词
%================================================================================
\keywordsEN{%
Keyword1;
Keyword2;
Keyword3;
Keyword4;
Keyword5
}%


%%
%% Copyright: Andy GAO (华工自动化)
%%
%%==========================================================
%%          华南理工大学博士学位论文——摘要
%%       Author: Andy GAO        Date: 2022.01.
%%==========================================================
%%


%% 目录
\tableofcontents
\addcontentsline{toc}{chapter}{目录}
% 【需手动修正目录页在“目录”中的页码】

%% 表格清单
\listoftables

%% 插图清单
\listoffigures



%================================================================================
%% 绪论,正文
%================================================================================
\mainmatter
%% 第一章:
%%
%% Copyright: Andy GAO (华工自动化)
%%
%%==========================================================
%%          华南理工大学博士学位论文——第一章
%%       Author: Andy GAO        Date: 2022.01.
%%==========================================================
%%

\chapter{文献引用、图片与表格}
\label{cha:Chapter1}


\section{文献引用示例}
\label{sec:1_Background}


\begin{comment}
%
华工自动化华工自动化华工自动化华工自动化华工自动化华工自动化华工自动化华工自动化华工自动化。

华工自动化华工自动化华工自动化华工自动化华工自动化华工自动化。
%
\end{comment}


研究背景及意义研究背景及意义{\citesp{lin2022scanning,bai2021hu,xu2021xin}}。
研究背景及意义研究背景及意义{\citesp{yan2021yi,li2019bo}},研究背景及意义研究背景及意义。

研究背景及意义(图{\ref{fig:1_Man&RobotOnGlass}\subref{fig:1_ManOnGlass}}{\citesp{hu2019ji}}),研究背景及意义,如图{\ref{fig:1_Man&RobotOnGlass}\subref{fig:1_RobotOnGlass}}所示{\citesp{url2015gao}}。


%
%%%%%%%%%%%%%%%%%%%%%%%%%%%%%%%%%%%%%%%%%%%%%%%%%%%%% Figure
\begin{figure}
  \centering
  \subfloat[]{ \label{fig:1_ManOnGlass}
  \includegraphics[width=0.45\linewidth,height=0.30\linewidth]{CHAPTER1/FIGs/tmp}}
  \qquad%\hfill
  \subfloat[]{ \label{fig:1_RobotOnGlass}
  \includegraphics[width=0.45\linewidth,height=0.30\linewidth]{CHAPTER1/FIGs/tmp}}
  \caption[高空玻璃幕墙环境中的(a)人工作业与(b)机器人作业方式]%图表目录的标题内容
  {高空玻璃幕墙环境中的(a)人工作业{\citesp{hu2019ji}}与(b)机器人作业方式{\citesp{url2015gao}}}
  \label{fig:1_Man&RobotOnGlass}
\end{figure}
%%%%%%%%%%%%%%%%%%%%%%%%%%%%%%%%%%%%%%%%%%%%%%%%%%%%% End Figure
%


文献{\cite{magazine2021,Nishi1986Design,alkalla2017tele,yoshida2021hangrawler,yang2016path}}研究背景及意义,
文献{\cite{zhao2018obstacle,yu2015ji}}研究背景及意义研究背景及意义{\citesp{yun2018ji,maciejowski2002predictive,hardy1988inequalities,jia2016ji}}。



\section{图片排版示例}
\label{sec:1_ResearchActuality}

研究现状研究现状。
研究现状研究现状。



\subsection{国内外研究现状}
\label{sec:1_ResearchStatus}

\subsubsection{图片排版}
\label{sec:1_BipedalClimbingRobot}


%
%%%%%%%%%%%%%%%%%%%%%%%%%%%%%%%%%%%%%%%%%%%%%%%%%%%%% Figure
\begin{figure}
  \centering
  \includegraphics[width=0.5\linewidth]{CHAPTER1/FIGs/tmp}
  \caption[外墙清洗双足机器人]%
  {外墙清洗双足机器人{\citesp{nansai2018design}}}
  \label{fig:1_FaçadeCleaningRobot}
\end{figure}
%%%%%%%%%%%%%%%%%%%%%%%%%%%%%%%%%%%%%%%%%%%%%%%%%%%%% End Figure
%


研究现状研究现状(如图{\ref{fig:1_FaçadeCleaningRobot}}所示),研究现状研究现状研究现状研究现状研究现状{\citesp{nansai2018design}}。
研究现状研究现状研究现状(如图{\ref{fig:1_iCrawl}}所示),研究现状研究现状研究现状{\citesp{khan2020icrawl}}。


%
%%%%%%%%%%%%%%%%%%%%%%%%%%%%%%%%%%%%%%%%%%%%%%%%%%%%% Figure Fig.
\begin{figure}
  \centering
\begin{minipage}{0.45\linewidth}
  \centering
  \includegraphics[width=\linewidth,height=0.60\linewidth]{CHAPTER1/FIGs/tmp}
  \caption[iCrawl机器人]%
  {iCrawl机器人{\citesp{khan2020icrawl}}}
  \label{fig:1_iCrawl}
\end{minipage}
\qquad%\hfill
\begin{minipage}{0.45\linewidth}
  \centering
  \includegraphics[width=\linewidth,height=0.60\linewidth]{CHAPTER1/FIGs/tmp}
  \caption[双足轮混合式攀爬机器人]%
  {双足轮混合式攀爬机器人{\citesp{jiang2014gait}}}
  \label{fig:1_BipedWheelHybridRobot}
\end{minipage}
\end{figure}
%%%%%%%%%%%%%%%%%%%%%%%%%%%%%%%%%%%%%%%%%%%%%%%%%%%%% End Figure
%


%
%%%%%%%%%%%%%%%%%%%%%%%%%%%%%%%%%%%%%%%%%%%%%%%%%%%%% Figure
\begin{figure}
  \centering
  \subfloat[吸盘型]{ \label{fig:1_SuctionBiped}
  \includegraphics[width=0.3\linewidth,height=0.25\linewidth]{CHAPTER1/FIGs/tmp}}
  \quad%\hfill
  \subfloat[夹爪型]{ \label{fig:1_GrippingBiped}
  \includegraphics[width=0.3\linewidth,height=0.25\linewidth]{CHAPTER1/FIGs/tmp}}
  \quad%\hfill
  \subfloat[软体型]{ \label{fig:1_SoftBiped}
  \includegraphics[width=0.3\linewidth,height=0.25\linewidth]{CHAPTER1/FIGs/tmp}}
  \caption[仿尺蠖型双足机器人]%
  {仿尺蠖型双足机器人{\citesp{fu2021shuang,guan2016climbot,zhang2017spatial}}}
  \label{fig:1_InchwormLikeBipedRobot}
\end{figure}
%%%%%%%%%%%%%%%%%%%%%%%%%%%%%%%%%%%%%%%%%%%%%%%%%%%%% End Figure
%


研究现状研究现状研究现状(图{\ref{fig:1_BipedWheelHybridRobot}}),研究现状研究现状{\citesp{jiang2014gait}}。
研究现状研究现状{\citesp{zhu2020planning,fu2021shuang}}
或研究现状研究{\citesp{guan2016climbot,gu2018optimal}}。
研究现状{\citesp{su2020fang,zhang2017spatial}}。
研究现状研究现状如图{\ref{fig:1_InchwormLikeBipedRobot}}所示。



\subsubsection{图片排版}
\label{sec:1_HexapodClimbingRobot}

现状研究现状研究现状研究现状,如图{\ref{fig:1_Wall&BridgeDetectionHexapod}}所示,研究现状研究现状研究现状研究现状{\citesp{cai2012ji,xie2015duo,ye2016degree,wei2016ji,ye2017ji}}。


%
%%%%%%%%%%%%%%%%%%%%%%%%%%%%%%%%%%%%%%%%%%%%%%%%%%%%% Figure
\begin{figure}
  \centering
  \subfloat{ \label{fig:1_WallDetectionHexapod}
  \includegraphics[width=0.45\linewidth,height=0.30\linewidth]{CHAPTER1/FIGs/tmp}}
  \qquad%\hfill
  \subfloat{ \label{fig:1_BridgeDetectionHexapod}
  \includegraphics[width=0.45\linewidth,height=0.30\linewidth]{CHAPTER1/FIGs/tmp}}
  \caption[用于墙体检测和桥梁检测的六足攀爬机器人]%
  {用于墙体检测和桥梁检测的六足攀爬机器人{\citesp{cai2012ji,wei2016ji}}}
  \label{fig:1_Wall&BridgeDetectionHexapod}
\end{figure}
%%%%%%%%%%%%%%%%%%%%%%%%%%%%%%%%%%%%%%%%%%%%%%%%%%%%% End Figure
%



\subsection{长表格、跨页表格示例}
\label{sec:1_LeggedClimbingRobotSummary}

长表格跨页表格长表格跨页表格长表格跨页表格长表格跨页表格长表格跨页表格长表格跨页表格长表格跨页表格长表格跨页表格长表格跨页表格长表格跨页表格长表格跨页表格长表格跨页表格长表格跨页表格长表格跨页表格长表格跨页表格长表格跨页表格长表格跨页表格长表格跨页表格长表格跨页表格长表格跨页表格长表格跨页表格长表格跨页表格长表格跨页表格长表格跨页表格长表格跨页表格长表格跨页表格长表格跨页表格长表格跨页表格长表格跨页表格长表格跨页表格长表格跨页表格长表格跨页表格长表格跨页表格长表格跨页表格长表格跨页表格。
汇总结果如表{\ref{tab:2_LeggedClimbingRobot}}所示。


%
%%%%%%%%%%%%%%%%%%%%%%%%%%%%%%%%%%%%%%%%%%%%%%%%%%%%% Table
\begin{longtable}{@{}m{7em}@{\hspace{0em}}>{\centering}m{2em}@{\hspace{1.5em}}m{5em}@{\hspace{1.5em}}>{\centering}m{4em}@{\hspace{1.5em}}m{5em}@{\hspace{1.5em}}m{7em}@{}}
\setstretch{1.5}\\
\caption{长表格跨页表格}
\label{tab:2_LeggedClimbingRobot}\\
\toprule[1.5pt]
{机器人名称}  &  {支链数目}  &  {机体形状}  &  {单腿主动副数目}  &  {末端装置}  &  \multicolumn{1}{c}{应用场景} \\
\hline
\endfirsthead
%\caption{典型足式攀爬机器人}\\
\multicolumn{6}{r}{\small 表{\ref{tab:2_LeggedClimbingRobot}}(续)}\\
\toprule[1.5pt]
{机器人名称}  &  {支链数目}  &  {机体形状}  &  {单腿主动副数目}  &  {末端装置}  &  \multicolumn{1}{c}{应用场景} \\
\hline
\endhead
\bottomrule[1.5pt]
\endfoot
\bottomrule[1.5pt]
\endlastfoot
{iCrawl}{\citesp{khan2020icrawl}}  &  {2}  &  {仿尺蠖型}  &  {2}  &  {电磁脚}  &  {金属外管道检测} \\
{CMBOT}  &  {2}  &  {仿尺蠖型}  &  {2}  &  {电磁模块}  &  {铁路桥梁检修} \\
{MRWALL-SPECT-IV}  &  {4}  &  {矩形}  &  {3}  &  {吸盘组}  &  {高层建筑维护} \\
{CLIBO}  &  {4}  &  {矩形}  &  {4}  &  {钩爪}  &  {攀岩} \\
{UNIclimb}  &  {4}  &  {矩形}  &  {3}  &  {干吸附足垫}  &  {防水爬行} \\
{Magneto}  &  {4}  &  {矩形}  &  {3}  &  {永久电磁体}  &  {铁质腐蚀检查} \\
{HubRobo}  &  {4}  &  {正方形}  &  {3}  &  {钩爪}  &  {火星探测} \\
{Nyxbot}  &  {4}  &  {矩形}  &  {4}  &  {干吸附圆盘}  &  {斜坡攀爬} \\
{SR-CR}  &  {4}  &  {弹性伸缩管}  &  {—}  &  {——}  &  {平行杆检测} \\
{ASTERISK}  &  {6}  &  {正六边形}  &  {4}  &  {半球形单元}  &  {网状天花板检修} \\
{DIGbot}  &  {6}  &  {附加关节的矩形}  &  {3}  &  {勾刺}  &  {壁—台翻越} \\
{RiSE}  &  {6}  &  {带尾矩形}  &  {2}  &  {钩爪}  &  {泥砖墙壁攀爬} \\
{Abigaille-III}  &  {6}  &  {正方形}  &  {3}  &  {仿生足垫}  &  {航天器检修} \\
{WALKMAN-I}  &  {6}  &  {柔性气动驱动器}  &  {—}  &  {真空吸盘}  &  {地—壁过渡} \\
{SURFY}  &  {8}  &  {内外框架式}  &  {—}  &  {真空吸盘}  &  {存储罐表面金属质量检测} \\
\end{longtable}
%%%%%%%%%%%%%%%%%%%%%%%%%%%%%%%%%%%%%%%%%%%%%%%%%%%%% End Table
%


典型对比典型对比典型对比典型对比典型典型对比典型对比典型对比典型对比典型对比典型对比对比(如表{\ref{tab:2_AdhesionType}}所示)。


%
%%%%%%%%%%%%%%%%%%%%%%%%%%%%%%%%%%%%%%%%%%%%%%%%%%%%% Table
\begin{longtable}{@{}c@{\hspace{2em}}m{10em}@{\hspace{2em}}m{10em}@{\hspace{2em}}m{6em}@{}}
\setstretch{1.5}\\
\caption{对比}
\label{tab:2_AdhesionType}\\
\toprule[1.5pt]
{吸附方式}  &  \multicolumn{1}{c}{\hspace{-1.5em}优点}  &  \multicolumn{1}{c}{\hspace{-1.5em}缺点}  &  \multicolumn{1}{c}{适用环境} \\
\hline
\endfirsthead
%\caption{常见吸附方式对比}\\
\multicolumn{4}{r}{\small 表{\ref{tab:2_AdhesionType}}(续)}\\
\toprule[1.5pt]
{吸附方式}  &  \multicolumn{1}{c}{\hspace{-1.5em}优点}  &  \multicolumn{1}{c}{\hspace{-1.5em}缺点}  &  \multicolumn{1}{c}{适用环境} \\
\hline
\endhead
\bottomrule[1.5pt]
\endfoot
\bottomrule[1.5pt]
\endlastfoot
{***吸附}  &  {可适应优点优点优点优点优点优点优点优点优点优点优点}  &  {缺点缺点缺点缺点缺点缺点缺点缺点}  &  {平整光滑的表面} \\
{***抓附}  &  {优点优点优点优点优点优点优}  &  {缺点缺点缺点缺点缺点缺点缺点}  &  {杆状建筑,具有孔洞裂缝的粗糙墙面} \\
{***吸附}  &  {优点优点优点优点优点优点}  &  {缺点缺点缺点缺点缺点缺点缺点}  &  {导磁材料壁面} \\
{***吸附}  &  {优点优点优点优点优点}  &  {缺点缺点缺点缺点缺点缺点}  &  {导磁材料壁面} \\
{***吸附}  &  {优点优点优点优点优点优点优点}  &  {缺点缺点缺点缺点缺点缺}  &  {较平整的表面} \\
\end{longtable}
%%%%%%%%%%%%%%%%%%%%%%%%%%%%%%%%%%%%%%%%%%%%%%%%%%%%% End Table
%


典型对比典型对比典型对比典型对比典型对比。



\section{关键科学技术问题概述}
\label{sec:1_ProblemOverview}

\textbf{(1)关键科学技术问题}

关键科学技术问题概述关键科学技术问题概述关键科学技术问题概述关键科学技术问题概述关键科学技术问题概述。

\textbf{(2)关键科学技术问题}

关键科学技术问题概述关键科学技术问题概述关键科学技术问题概述关键科学技术问题概述关键科学技术问题概述。



\section{本文的研究内容与组织结构}
\label{sec:1_ResearchContents&Organization}

具体来说,本文的主要研究内容如下。
各章之间的逻辑联系梳理如下。



\section{本章小结}
\label{sec:1_Conclusion}

本章主要。。。。。。


%%
%% Copyright: Andy GAO (华工自动化)
%%
%%==========================================================
%%          华南理工大学博士学位论文——第一章
%%       Author: Andy GAO        Date: 2022.01.
%%==========================================================
%%


%% 第二章:
%%
%% Copyright: Andy GAO (华工自动化)
%%
%%==========================================================
%%          华南理工大学博士学位论文——第二章
%%       Author: Andy GAO        Date: 2022.01.
%%==========================================================
%%

\chapter{数学公式-数学公式-数学公式-数学公式-数学公式-数学公式}
\label{cha:Chapter2}


\section{引言}
\label{sec:2_Introduction}

引言引言引言引言引言引言引言引言引言引言引言引言引言引言引言引言引言引言引言引言引言引言引言引言引言引言引言引言引言引言引言引言。



\section{数学公式}
\label{sec:2_MechanicalDesign}

机器人的机械结构设计机器人的机械结构设计分别如图{\ref{fig:2_PhysicalWelCH}}与图{\ref{fig:2_VirtualWelCH}}所示。


%
%%%%%%%%%%%%%%%%%%%%%%%%%%%%%%%%%%%%%%%%%%%%%%%%%%%%% Figure
\begin{figure}
  \centering
  \subfloat{
  \includegraphics[width=0.45\linewidth,height=0.30\linewidth]{CHAPTER2/FIGs/tmp} }
  \qquad%\hfill
  \subfloat{
  \includegraphics[width=0.45\linewidth,height=0.30\linewidth]{CHAPTER2/FIGs/tmp} }
  \caption{吸附式攀爬六足机器人WelCH的物理样机模型}
  \label{fig:2_PhysicalWelCH}
\end{figure}
%%%%%%%%%%%%%%%%%%%%%%%%%%%%%%%%%%%%%%%%%%%%%%%%%%%%% End Figure
%


%
%%%%%%%%%%%%%%%%%%%%%%%%%%%%%%%%%%%%%%%%%%%%%%%%%%%%% Figure
\begin{figure}
  \centering
  \includegraphics[width=0.5\linewidth]{CHAPTER2/FIGs/tmp}
  \caption{吸附式攀爬六足机器人WelCH的SolidWorks虚拟原型}
  \label{fig:2_VirtualWelCH}
\end{figure}
%%%%%%%%%%%%%%%%%%%%%%%%%%%%%%%%%%%%%%%%%%%%%%%%%%%%% End Figure
%



\subsection{数学公式数学公式}
\label{sec:2_StructureLayout}

机体与支链结构布局机体与支链结构布局,如图{\ref{fig:2_Body&Limb}\subref{fig:2_BodyOfWelCH}}所示。


%
%%%%%%%%%%%%%%%%%%%%%%%%%%%%%%%%%%%%%%%%%%%%%%%%%%%%% Figure
\begin{figure}[t]
  \centering
  \subfloat[]{ \label{fig:2_BodyOfWelCH}
  \includegraphics[width=0.35\linewidth,height=0.35\linewidth]{CHAPTER2/FIGs/tmp}}
  \qquad%\hfill
  \subfloat[]{ \label{fig:2_LimbOfWelCH}
  \includegraphics[width=0.35\linewidth,height=0.35\linewidth]{CHAPTER2/FIGs/tmp}}
  \caption{吸附式攀爬六足机器人WelCH的机体平台与单腿支链的结构布局}
  \label{fig:2_Body&Limb}
\end{figure}
%%%%%%%%%%%%%%%%%%%%%%%%%%%%%%%%%%%%%%%%%%%%%%%%%%%%% End Figure
%


机体与支链结构布局机体与支链结构布局(如图{\ref{fig:2_Body&Limb}\subref{fig:2_LimbOfWelCH}}所示)。
髋关节(记为关节1)的旋转轴线经过连接点$V_{\ell}$且垂直于机体平台平面。。。。。。(对应于关节3相对于关节2存在幅值为$90^{\circ}$ 的初始扭转角),该位形亦称为是六足机器人WelCH的{\it{标称位形}}。
机体与支链结构布局机体与支链结构布局如表{\ref{tab:2_JointRotationRange}}所列。


%
%%%%%%%%%%%%%%%%%%%%%%%%%%%%%%%%%%%%%%%%%%%%%%%%%%%%% Table
\begin{table}[t]
\setstretch{1.5}
\centering
\caption{单腿支链上四个主动关节的有效旋转范围}
\label{tab:2_JointRotationRange}
\begin{tabular}{ccc}
\toprule[1.5pt]
\multirow{2}{*}{旋转副}    &   \multicolumn{2}{c}{关节转动范围}   \\
                               \cline{2-3}
                           &   {角度制(${{}^\circ}$)}     &   {弧度制(${\rm rad}$)} \\
\hline
髋关节(关节1)            &   $[-60, 60]$               &   $[-\frac{\pi}{3}, \frac{\pi}{3}]$ \\
胫关节(关节2)            &   $[-90, 90]$               &   $[-\frac{\pi}{2}, \frac{\pi}{2}]$ \\
膝关节(关节3)            &   $[-45, 80]$               &   $[-\frac{\pi}{4}, \frac{4\pi}{9}]$ \\
踝关节(关节4)            &   $[-90, 90]$               &   $[-\frac{\pi}{2}, \frac{\pi}{2}]$ \\
\bottomrule[1.5pt]
\end{tabular}
\end{table}
%%%%%%%%%%%%%%%%%%%%%%%%%%%%%%%%%%%%%%%%%%%%%%%%%%%%% End Table
%



\subsection{数学公式数学公式数学公式}
\label{sec:2_HardwareDesign}

机器人的硬件系统设计机器人的硬件系统设计如图{\ref{fig:2_HardwareSystem}}所示。


%
%%%%%%%%%%%%%%%%%%%%%%%%%%%%%%%%%%%%%%%%%%%%%%%%%%%%% Figure
\begin{figure}
  \centering
  \includegraphics[width=0.9\linewidth]{CHAPTER2/FIGs/tmp}
  \caption{吸附式攀爬六足机器人WelCH的硬件系统架构图}
  \label{fig:2_HardwareSystem}
\end{figure}
%%%%%%%%%%%%%%%%%%%%%%%%%%%%%%%%%%%%%%%%%%%%%%%%%%%%% End Figure
%


机器人的硬件系统设计机器人的硬件系统设计机器人的硬件系统设计。
该型电机兼具转矩特性良好、效率较高、载荷较强、体量轻小等优点,其部分参数配置如表{\ref{tab:2_ServoMotorConfiguration}}所列。


%
%%%%%%%%%%%%%%%%%%%%%%%%%%%%%%%%%%%%%%%%%%%%%%%%%%%%% Table
\begin{table}
\setstretch{1.5}
\centering
\caption{DCX22L型有刷直流电机的参数配置表}
\label{tab:2_ServoMotorConfiguration}
\begin{tabular}{*{2}{c}@{\hspace{2.5em}}*{2}{c}}
\toprule[1.5pt]
条目                        &       数值            &       条目                            &       数值 \\
\hline
标称电压                    &       48 V            &       空载转速                        &       10100 rpm \\
空载电流                    &       16.2 mA         &       标称转速                        &       9020 rmp \\
标称转矩(最大连续转矩)    &       30.3 mNm        &       标称电流(最大连续电流)        &       0.687 A \\
失速转矩                    &       294 mNm         &       失速电流                        &       6.5 A \\
最大效率                    &       89.9{\%}        &       最大输出功率                    &       45.8 W \\
最大允许转速                &       18000 rpm       &       最大轴向载荷                    &       2.5 N \\
最大径向载荷                &       16 N            &       质量                            &       100 g \\
\bottomrule[1.5pt]
\end{tabular}
\end{table}
%%%%%%%%%%%%%%%%%%%%%%%%%%%%%%%%%%%%%%%%%%%%%%%%%%%%% End Table
%


机器人的硬件系统设计机器人的硬件系统设计。



\section{六足机器人的数学公式}
\label{sec:2_Kinematics}

六足机器人的运动学建模六足机器人的运动学建模六足机器人的运动学建模六足机器人的运动学建模六足机器人的运动学建模。


\subsection{六足机器人的数学公式数学公式}
\label{sec:2_ForwardKinematics}

六足机器人的正运动学建模如表{\ref{tab:2_PrototypeDimension}}所列。


%
%%%%%%%%%%%%%%%%%%%%%%%%%%%%%%%%%%%%%%%%%%%%%%%%%%%%% Table
\begin{table}
\setstretch{1.5}
\centering
\caption{机器人WelCH的物理样机的尺寸配置表}
\label{tab:2_PrototypeDimension}
\begin{tabular}{*{3}{c}@{\hspace{4em}}*{3}{c}}
\toprule[1.5pt]
条目  &  符号表示  &  数值  &  条目  &  符号表示  &  数值 \\
\hline
机体半径  &  $R_{\rm B}$  &  0.18{\:}m  &  吸盘半径  &  $r_{\rm S}$  &  0.06{\:}m \\
连杆1长度  &  $L_1$  &  0.09{\:}m  &  连杆2长度  &  $L_2$  &  0.15{\:}m \\
连杆3长度  &  $L_3$  &  0.16{\:}m  &  连杆4长度  &  $L_4$  &  0.15{\:}m \\
整机质量  &  $M$  &  25{\:}kg \\
\bottomrule[1.5pt]
\end{tabular}
\end{table}
%%%%%%%%%%%%%%%%%%%%%%%%%%%%%%%%%%%%%%%%%%%%%%%%%%%%% End Table
%


六足机器人的正运动学建模六足机器人的正运动学建模,相应的方向向量${\bm {\omega}}_j$($j = 1,2,3,4$)为
%
%%%%%%%%%%%%%%%%%%%%%%%%%%%%%%%%%%%%%%%%%%%%%%%%%%%%% Equation
\begin{equation}
{\bm {\omega}}_1
=
\begin{bmatrix}
0 \\ 0 \\ 1 \\
\end{bmatrix} ,
{\quad}
{\bm {\omega}}_2 = {\bm {\omega}}_3 = {\bm {\omega}}_4
=
\begin{bmatrix}
\sin{{\alpha}_{\ell}} \\
-\cos{{\alpha}_{\ell}} \\
0 \\
\end{bmatrix} .
\label{eq:2_AxisDirection}
\end{equation}
%%%%%%%%%%%%%%%%%%%%%%%%%%%%%%%%%%%%%%%%%%%%%%%%%%%%% End Equation
%

各关节轴线的位置矢量${\bm r}_j$可取为
%
%%%%%%%%%%%%%%%%%%%%%%%%%%%%%%%%%%%%%%%%%%%%%%%%%%%%% Equation
\begin{equation}
\begin{aligned}
{\bm r}_1
&=
\begin{bmatrix}
R_{\rm B} \cdot \cos{{\alpha}_{\ell}} \\
R_{\rm B} \cdot \sin{{\alpha}_{\ell}} \\
0 \\
\end{bmatrix} ,
{\quad}
%
{\bm r}_2
=
\begin{bmatrix}
(R_{\rm B} + L_1) \cdot \cos{{\alpha}_{\ell}} \\
(R_{\rm B} + L_1) \cdot \sin{{\alpha}_{\ell}} \\
0 \\
\end{bmatrix} ,
\\
%
{\bm r}_3
&=
\begin{bmatrix}
(R_{\rm B} + L_1 + L_2) \cdot \cos{{\alpha}_{\ell}} \\
(R_{\rm B} + L_1 + L_2) \cdot \sin{{\alpha}_{\ell}} \\
0 \\
\end{bmatrix} ,
{\quad}
%
{\bm r}_4
=
\begin{bmatrix}
(R_{\rm B} + L_1 + L_2) \cdot \cos{{\alpha}_{\ell}} \\
(R_{\rm B} + L_1 + L_2) \cdot \sin{{\alpha}_{\ell}} \\
-L_3 \\
\end{bmatrix} .
\end{aligned}
\label{eq:2_AxisPositionVector}
\end{equation}
%%%%%%%%%%%%%%%%%%%%%%%%%%%%%%%%%%%%%%%%%%%%%%%%%%%%% End Equation
%

则依据如下定义式:
%
%%%%%%%%%%%%%%%%%%%%%%%%%%%%%%%%%%%%%%%%%%%%%%%%%%%%% Equation
\begin{equation}
{\bm {\xi}}_j
{\:\triangleq\:}
\begin{bmatrix}
{\bm {\omega}}_j \\ {\bm v}_j \\
\end{bmatrix}
=
\begin{bmatrix}
{\bm {\omega}}_j \\
{\bm r}_j \times {\bm {\omega}}_j \\
\end{bmatrix} ,
\label{eq:2_ScrewRelationship}
\end{equation}
%%%%%%%%%%%%%%%%%%%%%%%%%%%%%%%%%%%%%%%%%%%%%%%%%%%%% End Equation
%
可得各关节的单位运动旋量${\bm {\xi}}_j$为
%
%%%%%%%%%%%%%%%%%%%%%%%%%%%%%%%%%%%%%%%%%%%%%%%%%%%%% Equation
\begin{equation}
\begin{aligned}
{\bm {\xi}}_1
&=
\begin{bmatrix}
0, 0, 1, R_{\rm B} \cdot \sin{{\alpha}_{\ell}}, -R_{\rm B} \cdot \cos{{\alpha}_{\ell}}, 0
\end{bmatrix}^{\rm T} ,
\\
%
{\bm {\xi}}_2
&=
\begin{bmatrix}
\sin{{\alpha}_{\ell}}, -\cos{{\alpha}_{\ell}}, 0, 0, 0, -(R_{\rm B} + L_1)
\end{bmatrix}^{\rm T} ,
\\
%
{\bm {\xi}}_3
&=
\begin{bmatrix}
\sin{{\alpha}_{\ell}}, -\cos{{\alpha}_{\ell}}, 0, 0, 0, -(R_{\rm B} + L_1 + L_2)
\end{bmatrix}^{\rm T} ,
\\
%
{\bm {\xi}}_4
&=
\begin{bmatrix}
\sin{{\alpha}_{\ell}}, -\cos{{\alpha}_{\ell}}, 0, -L_3 \cdot \cos{{\alpha}_{\ell}}, -L_3 \cdot \sin{{\alpha}_{\ell}}, -(R_{\rm B} + L_1 + L_2)
\end{bmatrix}^{\rm T} .
\end{aligned}
\label{eq:2_JointMotionScrew}
\end{equation}
%%%%%%%%%%%%%%%%%%%%%%%%%%%%%%%%%%%%%%%%%%%%%%%%%%%%% End Equation
%

而在初始标称位形下,足端系${\Sigma}^{{\rm F}_{\ell}}$相对于机体系${\Sigma}^{\rm B}$的齐次变换矩阵为
%
%%%%%%%%%%%%%%%%%%%%%%%%%%%%%%%%%%%%%%%%%%%%%%%%%%%%% Equation
\begin{equation}
\begin{aligned}
{\bm g}({\bm 0})
&=
{\rm Rot}(z,{\alpha}_{\ell}) \cdot {\rm Trans}(R_{\rm B},0,0) \cdot {\rm Rot}(x,\frac{\pi}{2}) \cdot {\rm Trans}(L_1+L_2,0,0) \\
&{\quad}
\cdot {\rm Rot}(z,-\frac{\pi}{2}) \cdot {\rm Trans}(L_3+L_4,0,0) \\
&=
\begin{bmatrix}
 0  &  \cos{{\alpha}_{\ell}}  &   \sin{{\alpha}_{\ell}}  &  (R_{\rm B} + L_1 + L_2) \cdot \cos{{\alpha}_{\ell}} \\
 0  &  \sin{{\alpha}_{\ell}}  &  -\cos{{\alpha}_{\ell}}  &  (R_{\rm B} + L_1 + L_2) \cdot \sin{{\alpha}_{\ell}} \\
-1  &  0                      &   0                      &  -(L_3+L_4) \\
 0  &  0                      &   0                      &  1 \\
\end{bmatrix} .
\end{aligned}
\label{eq:2_InitialTransformation}
\end{equation}
%%%%%%%%%%%%%%%%%%%%%%%%%%%%%%%%%%%%%%%%%%%%%%%%%%%%% End Equation
%
其中,${\rm Rot}({\kappa},{\phi})$表示绕着${\kappa}$轴旋转角度${\phi}$所对应的齐次旋转变换矩阵;
${\rm Trans}(t_x,t_y,t_z)$表示与向量${\bm t} {\:\triangleq\:} [t_x,t_y,t_z]^{\rm T}$对应的齐次平移变换矩阵。

具体来说,列向量${\bm t}_j = ({\bm I} - {\rm e}^{q_j \widehat{{\bm {\omega}}}_j})({\bm {\omega}}_j \times {\bm v}_j) + q_j {\bm {\omega}}_j {\bm {\omega}}_j^{\rm T} {\bm v}_j$($j = 1,2,3,4$)的表达式为
%
%%%%%%%%%%%%%%%%%%%%%%%%%%%%%%%%%%%%%%%%%%%%%%%%%%%%% Equation
\begin{equation*}
\begin{aligned}
{\bm t}_1
&=
\begin{bmatrix}
R_{\rm B} [ {\cos{{\alpha}_{\ell}}} - {\cos({\alpha}_{\ell} + q_1)} ] \\
R_{\rm B} [ {\sin{{\alpha}_{\ell}}} - {\sin({\alpha}_{\ell} + q_1)} ] \\
0
\end{bmatrix} ,
{\quad}
%
{\bm t}_2
=
\begin{bmatrix}
( R_{\rm B} + L_1 ) ( 1 - \cos{q_2} ) {\cos{{\alpha}_{\ell}}} \\
( R_{\rm B} + L_1 ) ( 1 - \cos{q_2} ) {\sin{{\alpha}_{\ell}}} \\
-( R_{\rm B} + L_1 ) \sin{q_2}
\end{bmatrix} ,
\\
%
{\bm t}_3
&=
\begin{bmatrix}
( R_{\rm B} + L_1 + L_2 ) ( 1 - \cos{q_3} ) {\cos{{\alpha}_{\ell}}} \\
( R_{\rm B} + L_1 + L_2 ) ( 1 - \cos{q_3} ) {\sin{{\alpha}_{\ell}}} \\
-( R_{\rm B} + L_1 + L_2 ) \sin{q_3}
\end{bmatrix} ,
\\
%
{\bm t}_4
&=
\begin{bmatrix}
[ ( R_{\rm B} + L_1 + L_2 ) ( 1 - \cos{q_4} ) - {L_3}{\sin{q_4}} ] {\cos{{\alpha}_{\ell}}} \\
[ ( R_{\rm B} + L_1 + L_2 ) ( 1 - \cos{q_4} ) - {L_3}{\sin{q_4}} ] {\sin{{\alpha}_{\ell}}} \\
-( R_{\rm B} + L_1 + L_2 ){\sin{q_4}} - {L_3}( 1 - \cos{q_4} )
\end{bmatrix} .
\end{aligned}
\label{eq:2_PositionInExponentialMap}
\end{equation*}
%%%%%%%%%%%%%%%%%%%%%%%%%%%%%%%%%%%%%%%%%%%%%%%%%%%%% End Equation
%

六足机器人的正运动学建模六足机器人的正运动学建模六足机器人的正运动学建模。。。。。。
其中,机体相对于各足端系${\Sigma}^{{\rm F}_{\ell}}$的瞬时位置坐标为
%
%%%%%%%%%%%%%%%%%%%%%%%%%%%%%%%%%%%%%%%%%%%%%%%%%%%%% Equation
\begin{equation}
\begin{cases}
\begin{aligned}
b_x
&=
-( R_{\rm B} \cdot \cos{q_1} + L_1 ) \cdot {\rm S}_{234} - L_2 \cdot {\rm S}_{34} - L_3 \cdot \cos{q_4} - L_4 , \\
%
b_y
&=
-( R_{\rm B} \cdot \cos{q_1} + L_1 ) \cdot {\rm C}_{234} - L_2 \cdot {\rm C}_{34} + L_3 \cdot \sin{q_4} , \\
%
b_z
&=
-R_{\rm B} \cdot \sin{q_1} . \\
\end{aligned}
\end{cases}
\label{eq:2_Body2FootPosition}
\end{equation}
%%%%%%%%%%%%%%%%%%%%%%%%%%%%%%%%%%%%%%%%%%%%%%%%%%%%% End Equation
%
上式中各符号的含义与前文一致。



\subsection{六足机器人的数学公式数学公式数学公式}
\label{sec:2_InverseKinematics}

六足机器人的逆运动学解算六足机器人的逆运动学解算。。。。。。
其中,${\rm sgn}(\cdot)$表示符号函数,即有:
$
{\rm sgn}(x) =
\begin{cases}
\begin{aligned}
&1 ,   &   \text{若} x > 0 , \\
&0 ,   &   \text{若} x = 0 , \\
&-1,   &   \text{若} x < 0 . \\
\end{aligned}
\end{cases}
$



\subsubsection{基于代数法的数学公式}
\label{sec:2_Algebraic2InverseKinematics}

\begin{lemma}
\label{lem:2_TrigonometricEquations}
基于代数法的逆运动学分析基于代数法的逆运动学分析基于代数法的逆运动学分析基于代数法的逆运动学分析。。。。。。
的解为
%
%%%%%%%%%%%%%%%%%%%%%%%%%%%%%%%%%%%%%%%%%%%%%%%%%%%%% Equation
\begin{equation}
\begin{cases}
\begin{aligned}
x &=
2 \cdot \arctan{ \frac{2AC \pm \sqrt{4 C^2 D^2 - (A^2 + B^2 - C^2 - D^2)^2}}{A^2 + (B + C)^2 - D^2} } ,\\
%
y &=
2 \cdot \arctan{ \frac{-2BD \pm \sqrt{4 C^2 D^2 - (A^2 + B^2 - C^2 - D^2)^2}}{(A + D)^2 + B^2 - C^2} } ,\\
\end{aligned}
\end{cases}
\label{eq:2_Solution4BinaryEquations}
\end{equation}
%%%%%%%%%%%%%%%%%%%%%%%%%%%%%%%%%%%%%%%%%%%%%%%%%%%%% End Equation
%
其中,$A$、$B$、$C$、$D$均为已知常数。
\end{lemma}


\begin{proof}
%%
基于代数法的逆运动学分析基于代数法的逆运动学分析基于代数法的逆运动学分析基于代数法的逆运动学分析基于代数法的逆运动学分析。
证毕。
%
\end{proof}


基于代数法的逆运动学分析基于代数法的逆运动学分析。。。。。。
%
%%%%%%%%%%%%%%%%%%%%%%%%%%%%%%%%%%%%%%%%%%%%%%%%%%%%% Equation
\begin{equation}
\begin{cases}
\begin{aligned}
A &{\;\triangleq\;}
p_z + L_4 \cdot \cos({\beta} - {\rm sgn}({\alpha}_{\ell}) \cdot {\varphi}) , \\
%
B &{\;\triangleq\;}
\frac{p_y - R_{\rm B} \cdot \sin{{\alpha}_{\ell}}}{\sin({\alpha}_{\ell} + q_1)} - L_1 - L_4 \cdot \sin({\beta} - {\rm sgn}({\alpha}_{\ell}) \cdot {\varphi}) \text{(若$\sin({\alpha}_{\ell} + q_1) \neq 0$)}, \\
& \text{或}{\;}
B {\;\triangleq\;}
\frac{p_x - R_{\rm B} \cdot \cos{{\alpha}_{\ell}}}{\cos({\alpha}_{\ell} + q_1)} - L_1 - L_4 \cdot \sin({\beta} - {\rm sgn}({\alpha}_{\ell}) \cdot {\varphi}) \text{(若$\cos({\alpha}_{\ell} + q_1) \neq 0$)}, \\
%
C &{\;\triangleq\;} L_2 ,
{\quad}
%
D {\;\triangleq\;} -L_3 ,
{\quad}
%
x {\;\triangleq\;} q_2 ,
{\quad}
%
y {\;\triangleq\;} q_2 + q_3 .
\end{aligned}
\end{cases}
\label{eq:2_DefineParameters}
\end{equation}
%%%%%%%%%%%%%%%%%%%%%%%%%%%%%%%%%%%%%%%%%%%%%%%%%%%%% End Equation
%

基于代数法的逆运动学分析基于代数法的逆运动学分析。



\subsubsection{基于几何法的数学公式}
\label{sec:2_Geometric2InverseKinematics}

基于几何法的逆运动学分析基于几何法的逆运动学分析基于几何法的逆运动学分析基于几何法的逆运动学分析基于几何法的逆运动学分析:
%
%%%%%%%%%%%%%%%%%%%%%%%%%%%%%%%%%%%%%%%%%%%%%%%%%%%%% Equation
\begin{equation}
\begin{cases}
\begin{aligned}
& \sin(q_2 - {\phi}) =
\frac{|D|}{\sqrt{A^2 + B^2}} \sin(q_3 + \frac{\pi}{2}) = \frac{|D| \cos{q_3}}{\sqrt{A^2 + B^2}} &{\quad} \text{(正弦定理)} \\
%
& \cos(q_2 - {\phi}) =
\frac{A^2 + B^2 + C^2 - D^2}{2|C| \sqrt{A^2 + B^2}} &{\quad} \text{(余弦定理)} \\
%
& {\phi} =
{\rm Atan2}(A, B) &{\quad} \text{(辅助角)} \\
\end{aligned}
\end{cases}
\label{eq:2_EquationsAboutJoint2}
\end{equation}
%%%%%%%%%%%%%%%%%%%%%%%%%%%%%%%%%%%%%%%%%%%%%%%%%%%%% End Equation
%

%
%%%%%%%%%%%%%%%%%%%%%%%%%%%%%%%%%%%%%%%%%%%%%%%%%%%%% Equation
\begin{equation*}
{\;\Longrightarrow\;}
\begin{cases}
\begin{aligned}
& \tan(q_2 - {\phi}) =
\frac{2|C||D| \cos{q_3}}{A^2 + B^2 + C^2 - D^2} \\
%
& {\phi} =
{\rm Atan2}(A, B) \\
\end{aligned}
\end{cases}
\end{equation*}
%%%%%%%%%%%%%%%%%%%%%%%%%%%%%%%%%%%%%%%%%%%%%%%%%%%%% End Equation
%
基于几何法的逆运动学分析基于几何法的逆运动学分析基于几何法的逆运动学分析基于几何法的逆运动学分析。


\begin{remark}
基于几何法的逆运动学分析基于几何法的逆运动学分析基于几何法的逆运动学分析基于几何法的逆运动学分析基于几何法的逆运动学分析。
\end{remark}




\section{注记与序号列表}
\label{sec:2_Jacobian}

速度雅可比矩阵分析速度雅可比矩阵分析速度雅可比矩阵分析速度雅可比矩阵分析。


\begin{remark}
关于速度雅可比矩阵分析速度雅可比矩阵分析速度雅可比矩阵分析速度雅可比矩阵分析速度雅可比矩阵分析如下:
\begin{compactenum}[\hspace{2em}(i)]
  \item
  注记与序号列表注记与序号列表注记与序号列表注记与序号列表注记与序号列表注记与序号列表;

  \item
  注记与序号列表;

  \item
  注记与序号列表注记与序号列表。
\end{compactenum}
\end{remark}



\section{本章小结}
\label{sec:2_Conclusion}

本章小结本章小结本章小结本章小结本章小结本章小结本章小结本章小结本章小结本章小结本章小结本章小结。


%%
%% Copyright: Andy GAO (华工自动化)
%%
%%==========================================================
%%          华南理工大学博士学位论文——第二章
%%       Author: Andy GAO        Date: 2022.01.
%%==========================================================
%%


%% 第三章:
%%
%% Copyright: Andy GAO (华工自动化)
%%
%%==========================================================
%%          华南理工大学博士学位论文——第三章
%%       Author: Andy GAO        Date: 2022.01.
%%==========================================================
%%

\chapter{引理、定理、证明、注记}
\label{cha:Chapter3}


\section{引言}
\label{sec:3_Introduction}

引言引言引言引言引言引言引言引言引言引言引言引言引言引言引言引言引言引言引言引言引言引言引言引言引言引言引言引言引言引言引言引言。



\section{定理}
\label{sec:3_BodyKinematics}

定理定理定理定理定理定理定理定理定理。

遵循下述定理~\ref{3_SlidingStability}中的结论,它的理论证明需借助引理~\ref{Lemma:3_PFS}中的结果。


\begin{lemma}
[{\cite{zhu2011attitude}}]
\label{Lemma:3_PFS}
对于一个非线性系统 $\dot{\bm x} = \bm{f(x, u)}$,假设存在一个正定函数 $V({\bm x})$ 满足 ${\dot V}({\bm x}) \leq - a V^r({\bm x}) + b$ (其中 $a>0$,$b>0$ 且 $0<r<1$),那么该系统的轨迹是有限时间实用稳定的(Practical Finite-time Stable,PFS)。
即该系统的状态能够在有限时间内收敛到域 $\{ {\bm x} \big{|} V^r({\bm x}) \leq \frac{b}{(1-\phi)a} \}$ 内,且稳定时间满足 $T \leq \frac{V^{1-r}({\bm x}(0))}{\phi(1-r)a}$($0 < \phi < 1$)。
\end{lemma}


\begin{theorem}
\label{3_SlidingStability}
稳定性。。。。。。
\end{theorem}


\begin{proof}
%%
定理定理定理定理定理定理定理定理定理定理定理定理定理定理定理定理定理定理定理定理定理。

证毕。
%%
% 【证毕(在\end{proof}之前不能有空行,否则证明结束符会错位!)】
\end{proof}


\begin{remark}
注记注记注记注记注记注记注记注记注记注记注记注记注记注记注记注记注记注记注记注记注记注记。
\end{remark}



\section{本章小结}
\label{sec:3_Conclusion}

本章小结本章小结本章小结本章小结本章小结本章小结本章小结本章小结本章小结本章小结本章小结本章小结本章小结本章小结本章小结本章小结。



%%
%% Copyright: Andy GAO (华工自动化)
%%
%%==========================================================
%%          华南理工大学博士学位论文——第三章
%%       Author: Andy GAO        Date: 2022.01.
%%==========================================================
%%


%% 第四章:
%%
%% Copyright: Andy GAO (华工自动化)
%%
%%==========================================================
%%          华南理工大学博士学位论文——第四章
%%       Author: Andy GAO        Date: 2022.01.
%%==========================================================
%%

\chapter{类定理格式的用法}
\label{cha:Chapter4}


\section{引言}
\label{sec:4_Introduction}

引言引言引言引言引言引言引言引言引言引言引言引言引言引言引言引言引言引言引言引言引言引言引言引言引言引言引言引言引言引言引言引言。



\section{约定}
\label{sec:4_AnalysisDOF}

自由度分析自由度分析自由度分析自由度分析自由度分析自由度分析。


\begin{convention}
\label{4_Convention1}
自由度分析自由度分析自由度分析自由度分析。
\end{convention}


\begin{convention}
\label{4_Convention2}
自由度分析自由度分析自由度分析自由度分析。
\end{convention}


%
%%%%%%%%%%%%%%%%%%%%%%%%%%%%%%%%%%%%%%%%%%%%%%%%%%%%% Table
\begin{table}
\centering
\setstretch{1.5}
\caption{机体实施俯仰运动过程中的跟踪控制性能对比}
\label{tab:4_ComparisonBody}
\setlength{\tabcolsep}{4mm}{
\begin{tabular}{ccccccc}
\toprule[1.5pt]
\multirow{2}{*}{足端序号}      & \multicolumn{3}{c}{SAE-PD}     & \multicolumn{3}{c}{SAE-ASMC} \\
                                \cmidrule(lr){2-4} \cmidrule(lr){5-7}
                                & $p_y$     & $p_z$     & $\rho$    & $p_y$     & $p_z$     & $\rho$ \\
\noalign{\smallskip}\hline\noalign{\smallskip}
足1    & 0.242     & 0.291     & 0.193     & 0.042     & 0.114     & 0.065 \\
足2    & 0.196     & 0.286     & 0.196     & 0.038     & 0.100     & 0.066 \\
足3    & 0.245     & 0.296     & 0.198     & 0.035     & 0.110     & 0.064 \\
足4    & 0.192     & 0.294     & 0.193     & 0.048     & 0.084     & 0.067 \\
足5    & 0.196     & 0.290     & 0.197     & 0.040     & 0.087     & 0.064 \\
足6    & 0.193     & 0.281     & 0.192     & 0.054     & 0.091     & 0.059 \\
\bottomrule[1.5pt]
\end{tabular}
}
\end{table}
%%%%%%%%%%%%%%%%%%%%%%%%%%%%%%%%%%%%%%%%%%%%%%%%%%%%% End Table
%


自由度分析自由度分析自由度分析自由度分析。



\section{本章小结}
\label{sec:4_Conclusion}

本章小结本章小结本章小结本章小结本章小结本章小结。



%%
%% Copyright: Andy GAO (华工自动化)
%%
%%==========================================================
%%          华南理工大学博士学位论文——第四章
%%       Author: Andy GAO        Date: 2022.01.
%%==========================================================
%%


%% 第五章:
%%
%% Copyright: Andy GAO (华工自动化)
%%
%%==========================================================
%%          华南理工大学博士学位论文——第五章
%%       Author: Andy GAO        Date: 2022.01.
%%==========================================================
%%

\chapter{算法-算法-算法-算法-算法-算法-算法-算法-算法-算法-算法}
\label{cha:Chapter5}


\section{引言}
\label{sec:5_Introduction}

引言引言引言引言引言引言引言引言。



\section{算法}
\label{sec:5_EnvironmentModelling}

算法算法算法算法算法,搜索过程如算法{\ref{alg:5_WorkspaceSearching}}所述。
%
%%%%%%%%%%%%%%%%%%%%%%%%%%%%%%%%%%%%%%%%%%%%%%%%%%%%% Algorithm
\begin{algorithm}[!ht]
\Indp               % 增大算法主体的缩进宽度
\SetInd{2em}{0.0em} % sets the size of the space before the vertical rule and after.
\LinesNumbered      % 显示行号
\setstretch{1.5}    % 设置行距为原行距的 1.5 倍
\caption{基于微元法搜索工作空间及其包络算法}
\label{alg:5_WorkspaceSearching}
%%--------------------------------------------------- 算法主体
  \KwIn{各关节转动范围,运动参数上下限(${\beta}_{\rm min}$,${\beta}_{\rm max}$,$pd_{\rm max}$)} % 符号\;输出分号并换行
  \For{吸盘俯仰角 ${\beta} := {\beta}_{\rm min}$ \KwTo ${\beta}_{\rm max}$}{
    \For{极角 ${\gamma} := -{\pi}$ \KwTo ${\pi}$}{
      \For{极径 $pd := 0$ \KwTo $pd_{\rm max}$}{
        {$x_{\rm f} = pd \cdot \cos{\gamma}$,$y_{\rm f} = pd \cdot \sin{\gamma}$;}\\
        {基于逆运动学方程计算各关节变量;}\\
        {检验各关节角度是否满足关节转动约束;}\\
        {记录有效的极径${\Omega}_{\rm p} \leftarrow pd$;}\\
      }
    \If{${\Omega}_{\rm p}$非空}{
      {记录当前有效的极角${\gamma}$;}\\
      {计算并记录有效的$x_{\rm f}$,$y_{\rm f}$,${\beta}$;}\\
      {寻找有效极径的最值;}%\\
      {计算并记录有效的$x_{\rm f}$与$y_{\rm f}$的最值;}\\
    }
    }
  }
  \KwOut{有效工作空间及其包络线}
%%---------------------------------------------------
\end{algorithm}
%%%%%%%%%%%%%%%%%%%%%%%%%%%%%%%%%%%%%%%%%%%%%%%%%%%%% End Algorithm
%


算法算法算法算法算法算法算法算法算法算法。



\section{本章小结}
\label{sec:5_Conclusion}

本章小结本章小结本章小结本章小结本章小结本章小结。



%%
%% Copyright: Andy GAO (华工自动化)
%%
%%==========================================================
%%          华南理工大学博士学位论文——第五章
%%       Author: Andy GAO        Date: 2022.01.
%%==========================================================
%%





%================================================================================
%% 结论,参考文献,(附录),研究成果,致谢
%================================================================================
\backmatter
%--------------------------------------------------------------------------------
%% 结论
%%
%% Copyright: Andy GAO (华工自动化)
%%
%%==========================================================
%%          华南理工大学博士学位论文——结论与展望
%%         Author: Andy GAO        Date: 2022.01.
%%==========================================================
%%

\chapterx{结论与展望}
\label{cha:Conclusion}

{\noindent\heiti\sanhao
1.{\quad}本文工作总结}
\label{sec:Conclusion}

本文工作总结本文工作总结本文工作总结本文工作总结本文工作总结。

(1)
本文工作总结本文工作总结本文工作总结本文工作总结本文工作总结。

(2)
本文工作总结本文工作总结本文工作总结本文工作总结。



\vspace{1.2em}
{\noindent\heiti\sanhao
2.{\quad}未来研究展望}
\label{sec:FutureWork}

未来研究展望未来研究展望未来研究展望未来研究展望未来研究展望。

(1)未来研究展望研究

未来研究展望未来研究展望未来研究展望未来研究展望未来研究展望。

(2)未来研究展望研究

未来研究展望未来研究展望未来研究展望未来研究展望未来研究展望。



%%
%% Copyright: Andy GAO (华工自动化)
%%
%%==========================================================
%%          华南理工大学博士学位论文——结论与展望
%%         Author: Andy GAO        Date: 2022.01.
%%==========================================================
%%



%--------------------------------------------------------------------------------
%% 参考文献
%\bibliographystyle{SCUTthesis_bibliographystyle}
%\bibliography{BibDatabase}
%%
%%
%% Copyright: Andy GAO (华工自动化)
%%
%%==========================================================
%%          华南理工大学博士学位论文——参考文献
%%        Author: Andy GAO        Date: 2022.04.
%%==========================================================
%%

\chapterxtrue
\chapterxname{参考文献}
\label{cha:References}


%%!!!!!!!!!!!!!!!!!!!!!!!!!!!!!!!!!!!!!!!!!!!!!!!!!!!!!!!!!!
%% 手动修改会议论文的年份格式——其后改为冒号:
%%!!!!!!!!!!!!!!!!!!!!!!!!!!!!!!!!!!!!!!!!!!!!!!!!!!!!!!!!!!


\begin{thebibliography}{152}
\setlength{\itemsep}{1ex}
\providecommand{\natexlab}[1]{#1}
\providecommand{\url}[1]{\texttt{#1}}
\expandafter\ifx\csname urlstyle\endcsname\relax
  \providecommand{\doi}[1]{doi: #1}\else
  \providecommand{\doi}{doi: \begingroup \urlstyle{rm}\Url}\fi

\bibitem[Lin et al.(2022)]{lin2022scanning}
Lin J., Hong X., Ren Z., {\it{et al}}.
\newblock Scanning laser in-depth heating infrared thermography for deep
  debonding of glass curtain walls structural adhesive[J].
\newblock Measurement, 2022, 192: 1--17.

\bibitem[白甲丽(2021)]{bai2021hu}
白甲丽.
\newblock 呼吸式玻璃幕墙建筑室内光热环境调控方法研究[D].
\newblock 西安: 西安建筑科技大学, 2021.

\bibitem[许钊(2021)]{xu2021xin}
许钊.
\newblock 新型玻璃幕墙光热电能量转化规律研究[D].
\newblock 呼和浩特: 内蒙古工业大学, 2021.

\bibitem[闫好萱~等(2021)]{yan2021yi}
闫好萱, 曹一帆, 冯博泉,等.
\newblock 异形玻璃幕墙清洗机器人设计研究[J].
\newblock 科技创新与应用, 2021, 11(13): 102--104+107.

\bibitem[李雪新~等(2019)]{li2019bo}
李雪新, 靳文停, 葛宜元,等.
\newblock 玻璃幕墙清洗机器人关键技术综述[J].
\newblock 木材加工机械, 2019, 30(6): 14--20.

\bibitem[胡郝君(2019)]{hu2019ji}
胡郝君.
\newblock
  基于涵道风扇的玻璃幕墙清洗机器人设计与姿态控制研究[D].
\newblock 哈尔滨: 哈尔滨工业大学, 2019.

\bibitem[url(2015)]{url2015gao}

\newblock 高楼清洁机器人[EB/OL].
  \url{https://www.zcool.com.cn/work/ZMTMxNDEyODg=.html}, 2015.

\bibitem[mag(2021年12月29日(第003版))]{magazine2021}

\newblock 《“十四五”机器人产业发展规划》解读[N].
  人民邮电, 2021年12月29日(第003版).

\bibitem[Nishi et al.(1986)]{Nishi1986Design}
Nishi A., Wakasugi Y., Watanabe K.
\newblock Design of a robot capable of moving on a vertical wall[J].
\newblock Advanced Robotics, 1986, 1(1): 33--45.

\bibitem[Alkalla et al.(2017)]{alkalla2017tele}
Alkalla M.G., Fanni M.A., Mohamed A.M., {\it{et al}}.
\newblock Tele-operated propeller-type climbing robot for inspection of
  petrochemical vessels[J].
\newblock Industrial Robot: An International Journal, 2017, 44(2): 166--177.

\bibitem[Yoshida et al.(2021)]{yoshida2021hangrawler}
Yoshida T., Yamada Y., Warisawa S., {\it{et al}}.
\newblock HanGrawler 2: Super-high-speed and large-payload ceiling mobile robot
  using crawler[A].
\newblock In: 2021 IEEE/RSJ International Conference on Intelligent Robots and
  Systems (IROS)[C].
\newblock Washington: IEEE, 2021: 2491--2497.

\bibitem[Yang et al.(2016)]{yang2016path}
Yang C.h.J., Paul G., Ward P., {\it{et al}}.
\newblock A path planning approach via task-objective pose selection with
  application to an inchworm-inspired climbing robot[A].
\newblock In: 2016 IEEE International Conference on Advanced Intelligent
  Mechatronics (AIM)[C].
\newblock Washington: IEEE, 2016: 401--406.

\bibitem[Zhao et al.(2018)]{zhao2018obstacle}
Zhao Y., Chai X., Gao F., {\it{et al}}.
\newblock Obstacle avoidance and motion planning scheme for a hexapod robot
  Octopus-III[J].
\newblock Robotics and Autonomous Systems, 2018, 103: 199--212.

\bibitem[于靖军~等(2015)]{yu2015ji}
于靖军, 刘辛军, 丁希仑.
\newblock 机器人机构学的数学基础[M].
\newblock 第2版.
\newblock 北京: 机械工业出版社, 2015.

\bibitem[贠超和王伟(2018)]{yun2018ji}
贠超, 王伟.
\newblock 机器人学导论[M].
\newblock 第4版.
\newblock 北京: 机械工业出版社, 2018.

\bibitem[Maciejowski(2002)]{maciejowski2002predictive}
Maciejowski J.M.
\newblock Predictive control: With constraints[M].
\newblock London: Pearson education, 2002.

\bibitem[Hardy et al.(1988)]{hardy1988inequalities}
Hardy G., Littlewood J., P{\'o}lya G.
\newblock {Inequalities}[M].
\newblock Cambridge Mathematical Library.
\newblock Cambridge: Cambridge University Press, 1988.

\bibitem[贾振中~等(2016)]{jia2016ji}
贾振中, 徐静, 付成龙,等.
\newblock 机器人建模和控制[M].
\newblock 北京: 机械工业出版社, 2016.

\bibitem[Nansai et al.(2018)]{nansai2018design}
Nansai S., Onodera K., Veerajagadheswar P., {\it{et al}}.
\newblock Design and experiment of a novel fa{\c{c}}ade cleaning robot with a
  biped mechanism[J].
\newblock Applied Sciences, 2018, 8(12): 1--17.

\bibitem[Khan et al.(2020)]{khan2020icrawl}
Khan M.B., Chuthong T., Do C.D., {\it{et al}}.
\newblock iCrawl: An inchworm-inspired crawling robot[J].
\newblock IEEE Access, 2020, 8: 200655--200668.

\bibitem[Jiang et al.(2014)]{jiang2014gait}
Jiang Y., Yue Z., Dong W., {\it{et al}}.
\newblock Gait planning of concave transition for a wall-climbing robot[A].
\newblock In: 2014 IEEE International Conference on Information and Automation
  (ICIA)[C].
\newblock Washington: IEEE, 2014: 1284--1288.

\bibitem[付紫杨(2021)]{fu2021shuang}
付紫杨.
\newblock 双足爬壁机器人空间定位与导航[D].
\newblock 广州: 广东工业大学, 2021.

\bibitem[Guan et al.(2016)]{guan2016climbot}
Guan Y., Jiang L., Zhu H., {\it{et al}}.
\newblock Climbot: A bio-inspired modular biped climbing robot—system
  development, climbing gaits, and experiments[J].
\newblock Journal of Mechanisms and Robotics, 2016, 8(2): 1--17.

\bibitem[Zhang et al.(2017)]{zhang2017spatial}
Zhang Y., Su M., Li M., {\it{et al}}.
\newblock A spatial soft module actuated by SMA coil[A].
\newblock In: 2017 IEEE International Conference on Mechatronics and Automation
  (ICMA)[C].
\newblock Washington: IEEE, 2017: 677--682.

\bibitem[Zhu et al.(2020)]{zhu2020planning}
Zhu H., Lu J., Gu S., {\it{et al}}.
\newblock Planning three-dimensional collision-free optimized climbing path for
  biped wall-climbing robots[J].
\newblock IEEE/ASME Transactions on Mechatronics, 2020, 26(5): 2712--2723.

\bibitem[Gu et al.(2018)]{gu2018optimal}
Gu S., Zhu H., Li H., {\it{et al}}.
\newblock Optimal collision-free grip planning for biped climbing robots in
  complex truss environment[J].
\newblock Applied Sciences, 2018, 8(12): 1--22.

\bibitem[苏满佳(2020)]{su2020fang}
苏满佳.
\newblock 仿生软体攀爬机器人的建模、分析与实验[D].
\newblock 广州: 广东工业大学, 2020.

\bibitem[蔡钊雄(2012)]{cai2012ji}
蔡钊雄.
\newblock
  基于多足爬墙机器人平台的桥梁裂缝检测方法研究[D].
\newblock 广州: 华南理工大学, 2012.

\bibitem[谢浩(2015)]{xie2015duo}
谢浩.
\newblock 多足爬墙机器人运动控制及步态规划研究[D].
\newblock 广州: 华南理工大学, 2015.

\bibitem[Ye et al.(2016)]{ye2016degree}
Ye C., Yuan Y., Wei W.
\newblock Degree of freedom analysis of hexapod wall-climbing robot[A].
\newblock In: 6th International Conference on Machinery, Materials,
  Environment, Biotechnology and Computer (MMEBC)[C].
\newblock Paris: Atlantis Press, 2016: 1041--1048.

\bibitem[魏武~等(2016)]{wei2016ji}
魏武, 叶春台, 袁银龙.
\newblock 基于群论的六足机器人运动空间研究[J].
\newblock 机器人, 2016, 38(5): 522--539.

\bibitem[叶春台(2017)]{ye2017ji}
叶春台.
\newblock
  基于螺旋理论与李群的六足机器人运动分析及步态规划[D].
\newblock 广州: 华南理工大学, 2017.

\bibitem[Zhu et al.(2011)]{zhu2011attitude}
Zhu Z., Xia Y., Fu M.
\newblock {Attitude stabilization of rigid spacecraft with finite-time
  convergence}[J].
\newblock International Journal of Robust and Nonlinear Control, 2011, 21(6):
  686--702.


\end{thebibliography}



%--------------------------------------------------------------------------------
%% 攻读博士学位期间取得的研究成果
%%
%% Completed: 2022/01/14
%% Copyright: Andy GAO (华工自动化)
%%
%%==========================================================
%%          攻读博士学位期间取得的研究成果
%%      Author: Andy GAO        Date: 2022.01.
%%==========================================================
%%

\chapterx{攻读博士学位期间取得的研究成果}

一、已发表(包括已接受待发表)的论文,以及已投稿、或已成文打算投稿、或拟成文投稿的论文情况
\textbf{\underline{(只填写与学位论文内容相关的部分):}}


\ifSingleBlindReview
%%----------------------------------------------------------
%% 单盲评审/终版提交格式
%%----------------------------------------------------------

%
%%%%%%%%%%%%%%%%%%%%%%%%%%%%%%%%%%%%%%%%%%%%%%%%%%%%% Table
\begin{longtable}{|>{\centering}m{1em}|>{\centering}m{5em}|>{\centering}m{8em}|>{\centering}m{6em}|>{\centering}m{4em}|>{\centering}m{4em}|>{\centering}m{2.9em}|}
%\setstretch{1.5}\\
\hline
{\textbf{序号}}  &  {\textbf{作者(全体作者,按顺序排列)}}  &  {\textbf{题{\qquad}目}}  &  {\textbf{发表或投稿刊物名称、级别}}  &  {\textbf{发表的卷期、年月、页码}}  &  {\textbf{与学位论文哪一部分(章、节)相关}}  &  {\textbf{被索引收录情况}}
%\\\hline
\endfirsthead
\hline
{\textbf{序号}}  &  {\textbf{作者(全体作者,按顺序排列)}}  &  {\textbf{题{\qquad}目}}  &  {\textbf{发表或投稿刊物名称、级别}}  &  {\textbf{发表的卷期、年月、页码}}  &  {\textbf{与学位论文哪一部分(章、节)相关}}  &  {\textbf{被索引收录情况}}
%\\\hline
\endhead
%\hline
\endfoot
%\hline
\endlastfoot
%
\hline
{1}  &  {\textbf{Yong Gao}, Wu Wei, Xinmei Wang, Dongliang Wang, Yanjie Li, Qiuda Yu}  &  {Paper Title}  &  {Information Sciences}  &  {2022, 606: 489--511}  &  {第三章,第六章}  &  {SCI, Q1 (IF= 8.233)}
\tabularnewline\hline
{2}  &  {\textbf{Yong Gao}, Wu Wei, Xinmei Wang, Yanjie Li, Dongliang Wang, Qiuda Yu}  &  {Paper Title}  &  {A}  &  {***}  &  {第五章}  &  {SCI, Q2 (IF= ***)}
\tabularnewline\hline
{3}  &  {\textbf{Yong Gao}, Wu Wei, Dongliang Wang, Zhun Fan, Xinmei Wang}  &  {Paper Title}  &  {IEEE}  &  {***}  &  {第四章}  &  {SCI, Q1 (IF= ***)}
\tabularnewline\hline
{4}  &  {\textbf{Yong Gao}, Dongliang Wang, Wu Wei, Qiuda Yu, Xiongding Liu, Yuhai Wei}  &  {Paper Title}  &  {D}  &  {***}  &  {第三章}  &  {SCI, Q2 (IF= ***)}
\tabularnewline\hline
{5}  &  {\textbf{Yong Gao}, Wu Wei, Yuhai Wei, Yanjie Li, Qiuda Yu, Dongliang Wang}  &  {Paper Title}  &  {N}  &  {拟投稿}  &  {第六章}  &  {SCI, Q1 (IF= ***)}
\tabularnewline\hline
%
\end{longtable}
%%%%%%%%%%%%%%%%%%%%%%%%%%%%%%%%%%%%%%%%%%%%%%%%%%%%% End Table
%





\newpage

%二、与学位内容相关的其它成果(包括专利、著作、获奖项目等)
二、博士期间获得的其它成果(包括专利、著作、获奖项目等)

{\noindent{\textbf{I、论文:}}}

\begin{compactenum}[\hspace{2em}(1)]
\item
Zhongbin Cai, {\textbf{Yong Gao*}}, Wu Wei, Tianxiao Gao, Zhijian Xie. Model Design ...... ....... Journal of ...... ......, 2021, 1754(1): 1--6. (EI, 通讯作者)

\item
Wenyu Xiao, {\textbf{Yong Gao*}}, Wu Wei, Jie Zhang, Xiaoman Tan. Adaptive ...... ....... Journal of ...... ......, 2021, 1754(1): 1--6. (EI, 通讯作者)
\end{compactenum}



{\noindent{\textbf{II、专利:}}}

\begin{compactenum}[\hspace{2em}(1)]
\item
{\textbf{高勇}},魏武,蔡中斌. 一种。。。。。。,ZL 2020 1 1435837.0. 发明专利,{\textbf{审定授权}}. 授权公告日:2022.04.22.

\item
周翔,{\textbf{高勇}},魏武,蔡中斌. 一种。。。。。。,ZL 2020 1 1435802.7. 发明专利,{\textbf{审定授权}}. 授权公告日:2022.08.16.
\end{compactenum}



{\noindent{\textbf{III、参与项目:}}}

\begin{compactenum}[\hspace{2em}(1)]
\item
广东省科技计划项目:建筑墙体。。。。。。(2015B。。。。。。),2015.06--2018.05,经费:500万

\item
国家自然科学基金面上项目:基于。。。。。。(61。。。。。。),2016.01--2019.12,经费:74.8万

\item
广东省产学研项目:人。。。。。。(2019A。。。。。。),2019.11--2022.10,经费:100万
\end{compactenum}



{\noindent{\textbf{IV、获奖竞赛:}}}

\begin{compactenum}[\hspace{2em}(1)]
\item
。。。。。。大赛,三等奖,2019

\item
。。。。。。大赛,一等奖,2020

\item
。。。。。。大赛,一等奖,2021
\end{compactenum}





\else
%%----------------------------------------------------------
%% 双盲评审格式
%%----------------------------------------------------------

%
%%%%%%%%%%%%%%%%%%%%%%%%%%%%%%%%%%%%%%%%%%%%%%%%%%%%% Table
\begin{longtable}{|>{\centering}m{1em}|>{\centering}m{12em}|>{\centering}m{6em}|>{\centering}m{4em}|>{\centering}m{6em}|>{\centering}m{3em}|}
%\setstretch{1.5}\\
\hline
{\textbf{序号}}  &  {\textbf{发表或投稿刊物/会议名称}}  &  {\textbf{作者(仅注明第几作者)}}  &  {\textbf{发表年份}}  &  {\textbf{与学位论文哪一部分(章、节)相关}}  &  {\textbf{被索引收录情况}}
%\\\hline
\endfirsthead
\hline
{\textbf{序号}}  &  {\textbf{发表或投稿刊物/会议名称}}  &  {\textbf{作者(仅注明第几作者)}}  &  {\textbf{发表年份}}  &  {\textbf{与学位论文哪一部分(章、节)相关}}  &  {\textbf{被索引收录情况}}
%\\\hline
\endhead
%\hline
\endfoot
%\hline
\endlastfoot
%
\hline
{1}  &  {华南}  &  {第一作者}  &  {2020}  &  {第二章}  &  {EI}
\tabularnewline\hline
{2}  &  {Information Sciences}  &  {第一作者}  &  {2021}  &  {第三章,第六章}  &  {SCI, Q1 (IF= 8.233)}
\tabularnewline\hline
{3}  &  {A}  &  {第一作者}  &  {2022}  &  {第三章}  &  {SCI, Q1 (IF= ***)}
\tabularnewline\hline
{4}  &  {IEEE}  &  {第一作者}  &  {2022}  &  {第四章}  &  {SCI, Q1 (IF= ***)}
\tabularnewline\hline
{5}  &  {A}  &  {第一作者}  &  {2021}  &  {第五章}  &  {SCI, Q1 (IF= ***)}
\tabularnewline\hline
{6}  &  {N}  &  {第一作者}  &  {拟投稿}  &  {第六章}  &  {SCI, Q1 (IF= ***)}
\tabularnewline\hline
%
\end{longtable}
%%%%%%%%%%%%%%%%%%%%%%%%%%%%%%%%%%%%%%%%%%%%%%%%%%%%% End Table
%



\newpage

二、与学位内容相关的其它成果(包括专利、著作、获奖项目等)

{\noindent{\textbf{I、专利:}}}

\begin{compactenum}[\hspace{2em}(1)]
\item
已授权一项发明专利,第一发明人,2022

\item
已受理一项发明专利,第一发明人,2021
\end{compactenum}



{\noindent{\textbf{II、获奖项目:}}}

\begin{compactenum}[\hspace{2em}(1)]
\item
获中国研究生系列竞赛项目一等奖一项,第一获奖人,2020

\item
获中国研究生系列竞赛项目三等奖一项,第一获奖人,2019
\end{compactenum}





\fi



\vfill



%--------------------------------------------------------------------------------
%% 致谢
%%
%% Copyright: Andy GAO (华工自动化)
%%
%%==========================================================
%%          华南理工大学博士学位论文——致谢
%%       Author: Andy GAO        Date: 2022.01.
%%==========================================================
%%

\chapterx{致谢}
\label{cha:Acknowledgements}


\ifSingleBlindReview
%%----------------------------------------------------------
%% 单盲评审/终版提交格式
%%----------------------------------------------------------


%\begin{comment}

五载韶华疾渡,万千思绪纷飞。
致谢致谢致谢致谢致谢致谢致谢致谢致谢致谢致谢致谢致谢致谢致谢致谢致谢致谢致谢致谢致谢致谢致谢。

%\end{comment}


\begin{flushright}
\makebox[10em][c]{高{\quad}勇}
\par
\makebox[10em][c]{2022年6月于华园湖心亭}
\end{flushright}



\fi

\vfill



%%
%% Copyright: Andy GAO (华工自动化)
%%
%%==========================================================
%%          华南理工大学博士学位论文——致谢
%%       Author: Andy GAO        Date: 2022.01.
%%==========================================================
%%


\end{document}

%%
%% Copyright: Andy GAO (华工自动化)
%%
%%==========================================================
%%            华南理工大学博士学位论文
%%      Author: Andy GAO        Date: 2022.01.
%%==========================================================
%%
