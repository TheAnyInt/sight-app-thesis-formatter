%%
%% Copyright: Andy GAO (华工自动化)
%%
%%==========================================================
%%          华南理工大学博士学位论文——第一章
%%       Author: Andy GAO        Date: 2022.01.
%%==========================================================
%%

\chapter{文献引用、图片与表格}
\label{cha:Chapter1}


\section{文献引用示例}
\label{sec:1_Background}


\begin{comment}
%
华工自动化华工自动化华工自动化华工自动化华工自动化华工自动化华工自动化华工自动化华工自动化。

华工自动化华工自动化华工自动化华工自动化华工自动化华工自动化。
%
\end{comment}


研究背景及意义研究背景及意义{\citesp{lin2022scanning,bai2021hu,xu2021xin}}。
研究背景及意义研究背景及意义{\citesp{yan2021yi,li2019bo}},研究背景及意义研究背景及意义。

研究背景及意义(图{\ref{fig:1_Man&RobotOnGlass}\subref{fig:1_ManOnGlass}}{\citesp{hu2019ji}}),研究背景及意义,如图{\ref{fig:1_Man&RobotOnGlass}\subref{fig:1_RobotOnGlass}}所示{\citesp{url2015gao}}。


%
%%%%%%%%%%%%%%%%%%%%%%%%%%%%%%%%%%%%%%%%%%%%%%%%%%%%% Figure
\begin{figure}
  \centering
  \subfloat[]{ \label{fig:1_ManOnGlass}
  \includegraphics[width=0.45\linewidth,height=0.30\linewidth]{CHAPTER1/FIGs/tmp}}
  \qquad%\hfill
  \subfloat[]{ \label{fig:1_RobotOnGlass}
  \includegraphics[width=0.45\linewidth,height=0.30\linewidth]{CHAPTER1/FIGs/tmp}}
  \caption[高空玻璃幕墙环境中的(a)人工作业与(b)机器人作业方式]%图表目录的标题内容
  {高空玻璃幕墙环境中的(a)人工作业{\citesp{hu2019ji}}与(b)机器人作业方式{\citesp{url2015gao}}}
  \label{fig:1_Man&RobotOnGlass}
\end{figure}
%%%%%%%%%%%%%%%%%%%%%%%%%%%%%%%%%%%%%%%%%%%%%%%%%%%%% End Figure
%


文献{\cite{magazine2021,Nishi1986Design,alkalla2017tele,yoshida2021hangrawler,yang2016path}}研究背景及意义,
文献{\cite{zhao2018obstacle,yu2015ji}}研究背景及意义研究背景及意义{\citesp{yun2018ji,maciejowski2002predictive,hardy1988inequalities,jia2016ji}}。



\section{图片排版示例}
\label{sec:1_ResearchActuality}

研究现状研究现状。
研究现状研究现状。



\subsection{国内外研究现状}
\label{sec:1_ResearchStatus}

\subsubsection{图片排版}
\label{sec:1_BipedalClimbingRobot}


%
%%%%%%%%%%%%%%%%%%%%%%%%%%%%%%%%%%%%%%%%%%%%%%%%%%%%% Figure
\begin{figure}
  \centering
  \includegraphics[width=0.5\linewidth]{CHAPTER1/FIGs/tmp}
  \caption[外墙清洗双足机器人]%
  {外墙清洗双足机器人{\citesp{nansai2018design}}}
  \label{fig:1_FaçadeCleaningRobot}
\end{figure}
%%%%%%%%%%%%%%%%%%%%%%%%%%%%%%%%%%%%%%%%%%%%%%%%%%%%% End Figure
%


研究现状研究现状(如图{\ref{fig:1_FaçadeCleaningRobot}}所示),研究现状研究现状研究现状研究现状研究现状{\citesp{nansai2018design}}。
研究现状研究现状研究现状(如图{\ref{fig:1_iCrawl}}所示),研究现状研究现状研究现状{\citesp{khan2020icrawl}}。


%
%%%%%%%%%%%%%%%%%%%%%%%%%%%%%%%%%%%%%%%%%%%%%%%%%%%%% Figure Fig.
\begin{figure}
  \centering
\begin{minipage}{0.45\linewidth}
  \centering
  \includegraphics[width=\linewidth,height=0.60\linewidth]{CHAPTER1/FIGs/tmp}
  \caption[iCrawl机器人]%
  {iCrawl机器人{\citesp{khan2020icrawl}}}
  \label{fig:1_iCrawl}
\end{minipage}
\qquad%\hfill
\begin{minipage}{0.45\linewidth}
  \centering
  \includegraphics[width=\linewidth,height=0.60\linewidth]{CHAPTER1/FIGs/tmp}
  \caption[双足轮混合式攀爬机器人]%
  {双足轮混合式攀爬机器人{\citesp{jiang2014gait}}}
  \label{fig:1_BipedWheelHybridRobot}
\end{minipage}
\end{figure}
%%%%%%%%%%%%%%%%%%%%%%%%%%%%%%%%%%%%%%%%%%%%%%%%%%%%% End Figure
%


%
%%%%%%%%%%%%%%%%%%%%%%%%%%%%%%%%%%%%%%%%%%%%%%%%%%%%% Figure
\begin{figure}
  \centering
  \subfloat[吸盘型]{ \label{fig:1_SuctionBiped}
  \includegraphics[width=0.3\linewidth,height=0.25\linewidth]{CHAPTER1/FIGs/tmp}}
  \quad%\hfill
  \subfloat[夹爪型]{ \label{fig:1_GrippingBiped}
  \includegraphics[width=0.3\linewidth,height=0.25\linewidth]{CHAPTER1/FIGs/tmp}}
  \quad%\hfill
  \subfloat[软体型]{ \label{fig:1_SoftBiped}
  \includegraphics[width=0.3\linewidth,height=0.25\linewidth]{CHAPTER1/FIGs/tmp}}
  \caption[仿尺蠖型双足机器人]%
  {仿尺蠖型双足机器人{\citesp{fu2021shuang,guan2016climbot,zhang2017spatial}}}
  \label{fig:1_InchwormLikeBipedRobot}
\end{figure}
%%%%%%%%%%%%%%%%%%%%%%%%%%%%%%%%%%%%%%%%%%%%%%%%%%%%% End Figure
%


研究现状研究现状研究现状(图{\ref{fig:1_BipedWheelHybridRobot}}),研究现状研究现状{\citesp{jiang2014gait}}。
研究现状研究现状{\citesp{zhu2020planning,fu2021shuang}}
或研究现状研究{\citesp{guan2016climbot,gu2018optimal}}。
研究现状{\citesp{su2020fang,zhang2017spatial}}。
研究现状研究现状如图{\ref{fig:1_InchwormLikeBipedRobot}}所示。



\subsubsection{图片排版}
\label{sec:1_HexapodClimbingRobot}

现状研究现状研究现状研究现状,如图{\ref{fig:1_Wall&BridgeDetectionHexapod}}所示,研究现状研究现状研究现状研究现状{\citesp{cai2012ji,xie2015duo,ye2016degree,wei2016ji,ye2017ji}}。


%
%%%%%%%%%%%%%%%%%%%%%%%%%%%%%%%%%%%%%%%%%%%%%%%%%%%%% Figure
\begin{figure}
  \centering
  \subfloat{ \label{fig:1_WallDetectionHexapod}
  \includegraphics[width=0.45\linewidth,height=0.30\linewidth]{CHAPTER1/FIGs/tmp}}
  \qquad%\hfill
  \subfloat{ \label{fig:1_BridgeDetectionHexapod}
  \includegraphics[width=0.45\linewidth,height=0.30\linewidth]{CHAPTER1/FIGs/tmp}}
  \caption[用于墙体检测和桥梁检测的六足攀爬机器人]%
  {用于墙体检测和桥梁检测的六足攀爬机器人{\citesp{cai2012ji,wei2016ji}}}
  \label{fig:1_Wall&BridgeDetectionHexapod}
\end{figure}
%%%%%%%%%%%%%%%%%%%%%%%%%%%%%%%%%%%%%%%%%%%%%%%%%%%%% End Figure
%



\subsection{长表格、跨页表格示例}
\label{sec:1_LeggedClimbingRobotSummary}

长表格跨页表格长表格跨页表格长表格跨页表格长表格跨页表格长表格跨页表格长表格跨页表格长表格跨页表格长表格跨页表格长表格跨页表格长表格跨页表格长表格跨页表格长表格跨页表格长表格跨页表格长表格跨页表格长表格跨页表格长表格跨页表格长表格跨页表格长表格跨页表格长表格跨页表格长表格跨页表格长表格跨页表格长表格跨页表格长表格跨页表格长表格跨页表格长表格跨页表格长表格跨页表格长表格跨页表格长表格跨页表格长表格跨页表格长表格跨页表格长表格跨页表格长表格跨页表格长表格跨页表格长表格跨页表格长表格跨页表格。
汇总结果如表{\ref{tab:2_LeggedClimbingRobot}}所示。


%
%%%%%%%%%%%%%%%%%%%%%%%%%%%%%%%%%%%%%%%%%%%%%%%%%%%%% Table
\begin{longtable}{@{}m{7em}@{\hspace{0em}}>{\centering}m{2em}@{\hspace{1.5em}}m{5em}@{\hspace{1.5em}}>{\centering}m{4em}@{\hspace{1.5em}}m{5em}@{\hspace{1.5em}}m{7em}@{}}
\setstretch{1.5}\\
\caption{长表格跨页表格}
\label{tab:2_LeggedClimbingRobot}\\
\toprule[1.5pt]
{机器人名称}  &  {支链数目}  &  {机体形状}  &  {单腿主动副数目}  &  {末端装置}  &  \multicolumn{1}{c}{应用场景} \\
\hline
\endfirsthead
%\caption{典型足式攀爬机器人}\\
\multicolumn{6}{r}{\small 表{\ref{tab:2_LeggedClimbingRobot}}(续)}\\
\toprule[1.5pt]
{机器人名称}  &  {支链数目}  &  {机体形状}  &  {单腿主动副数目}  &  {末端装置}  &  \multicolumn{1}{c}{应用场景} \\
\hline
\endhead
\bottomrule[1.5pt]
\endfoot
\bottomrule[1.5pt]
\endlastfoot
{iCrawl}{\citesp{khan2020icrawl}}  &  {2}  &  {仿尺蠖型}  &  {2}  &  {电磁脚}  &  {金属外管道检测} \\
{CMBOT}  &  {2}  &  {仿尺蠖型}  &  {2}  &  {电磁模块}  &  {铁路桥梁检修} \\
{MRWALL-SPECT-IV}  &  {4}  &  {矩形}  &  {3}  &  {吸盘组}  &  {高层建筑维护} \\
{CLIBO}  &  {4}  &  {矩形}  &  {4}  &  {钩爪}  &  {攀岩} \\
{UNIclimb}  &  {4}  &  {矩形}  &  {3}  &  {干吸附足垫}  &  {防水爬行} \\
{Magneto}  &  {4}  &  {矩形}  &  {3}  &  {永久电磁体}  &  {铁质腐蚀检查} \\
{HubRobo}  &  {4}  &  {正方形}  &  {3}  &  {钩爪}  &  {火星探测} \\
{Nyxbot}  &  {4}  &  {矩形}  &  {4}  &  {干吸附圆盘}  &  {斜坡攀爬} \\
{SR-CR}  &  {4}  &  {弹性伸缩管}  &  {—}  &  {——}  &  {平行杆检测} \\
{ASTERISK}  &  {6}  &  {正六边形}  &  {4}  &  {半球形单元}  &  {网状天花板检修} \\
{DIGbot}  &  {6}  &  {附加关节的矩形}  &  {3}  &  {勾刺}  &  {壁—台翻越} \\
{RiSE}  &  {6}  &  {带尾矩形}  &  {2}  &  {钩爪}  &  {泥砖墙壁攀爬} \\
{Abigaille-III}  &  {6}  &  {正方形}  &  {3}  &  {仿生足垫}  &  {航天器检修} \\
{WALKMAN-I}  &  {6}  &  {柔性气动驱动器}  &  {—}  &  {真空吸盘}  &  {地—壁过渡} \\
{SURFY}  &  {8}  &  {内外框架式}  &  {—}  &  {真空吸盘}  &  {存储罐表面金属质量检测} \\
\end{longtable}
%%%%%%%%%%%%%%%%%%%%%%%%%%%%%%%%%%%%%%%%%%%%%%%%%%%%% End Table
%


典型对比典型对比典型对比典型对比典型典型对比典型对比典型对比典型对比典型对比典型对比对比(如表{\ref{tab:2_AdhesionType}}所示)。


%
%%%%%%%%%%%%%%%%%%%%%%%%%%%%%%%%%%%%%%%%%%%%%%%%%%%%% Table
\begin{longtable}{@{}c@{\hspace{2em}}m{10em}@{\hspace{2em}}m{10em}@{\hspace{2em}}m{6em}@{}}
\setstretch{1.5}\\
\caption{对比}
\label{tab:2_AdhesionType}\\
\toprule[1.5pt]
{吸附方式}  &  \multicolumn{1}{c}{\hspace{-1.5em}优点}  &  \multicolumn{1}{c}{\hspace{-1.5em}缺点}  &  \multicolumn{1}{c}{适用环境} \\
\hline
\endfirsthead
%\caption{常见吸附方式对比}\\
\multicolumn{4}{r}{\small 表{\ref{tab:2_AdhesionType}}(续)}\\
\toprule[1.5pt]
{吸附方式}  &  \multicolumn{1}{c}{\hspace{-1.5em}优点}  &  \multicolumn{1}{c}{\hspace{-1.5em}缺点}  &  \multicolumn{1}{c}{适用环境} \\
\hline
\endhead
\bottomrule[1.5pt]
\endfoot
\bottomrule[1.5pt]
\endlastfoot
{***吸附}  &  {可适应优点优点优点优点优点优点优点优点优点优点优点}  &  {缺点缺点缺点缺点缺点缺点缺点缺点}  &  {平整光滑的表面} \\
{***抓附}  &  {优点优点优点优点优点优点优}  &  {缺点缺点缺点缺点缺点缺点缺点}  &  {杆状建筑,具有孔洞裂缝的粗糙墙面} \\
{***吸附}  &  {优点优点优点优点优点优点}  &  {缺点缺点缺点缺点缺点缺点缺点}  &  {导磁材料壁面} \\
{***吸附}  &  {优点优点优点优点优点}  &  {缺点缺点缺点缺点缺点缺点}  &  {导磁材料壁面} \\
{***吸附}  &  {优点优点优点优点优点优点优点}  &  {缺点缺点缺点缺点缺点缺}  &  {较平整的表面} \\
\end{longtable}
%%%%%%%%%%%%%%%%%%%%%%%%%%%%%%%%%%%%%%%%%%%%%%%%%%%%% End Table
%


典型对比典型对比典型对比典型对比典型对比。



\section{关键科学技术问题概述}
\label{sec:1_ProblemOverview}

\textbf{(1)关键科学技术问题}

关键科学技术问题概述关键科学技术问题概述关键科学技术问题概述关键科学技术问题概述关键科学技术问题概述。

\textbf{(2)关键科学技术问题}

关键科学技术问题概述关键科学技术问题概述关键科学技术问题概述关键科学技术问题概述关键科学技术问题概述。



\section{本文的研究内容与组织结构}
\label{sec:1_ResearchContents&Organization}

具体来说,本文的主要研究内容如下。
各章之间的逻辑联系梳理如下。



\section{本章小结}
\label{sec:1_Conclusion}

本章主要。。。。。。


%%
%% Copyright: Andy GAO (华工自动化)
%%
%%==========================================================
%%          华南理工大学博士学位论文——第一章
%%       Author: Andy GAO        Date: 2022.01.
%%==========================================================
%%
