%%
%% Copyright: Andy GAO (华工自动化)
%%
%%==========================================================
%%          华南理工大学博士学位论文——参考文献
%%        Author: Andy GAO        Date: 2022.04.
%%==========================================================
%%

\chapterxtrue
\chapterxname{参考文献}
\label{cha:References}


%%!!!!!!!!!!!!!!!!!!!!!!!!!!!!!!!!!!!!!!!!!!!!!!!!!!!!!!!!!!
%% 手动修改会议论文的年份格式——其后改为冒号:
%%!!!!!!!!!!!!!!!!!!!!!!!!!!!!!!!!!!!!!!!!!!!!!!!!!!!!!!!!!!


\begin{thebibliography}{152}
\setlength{\itemsep}{1ex}
\providecommand{\natexlab}[1]{#1}
\providecommand{\url}[1]{\texttt{#1}}
\expandafter\ifx\csname urlstyle\endcsname\relax
  \providecommand{\doi}[1]{doi: #1}\else
  \providecommand{\doi}{doi: \begingroup \urlstyle{rm}\Url}\fi

\bibitem[Lin et al.(2022)]{lin2022scanning}
Lin J., Hong X., Ren Z., {\it{et al}}.
\newblock Scanning laser in-depth heating infrared thermography for deep
  debonding of glass curtain walls structural adhesive[J].
\newblock Measurement, 2022, 192: 1--17.

\bibitem[白甲丽(2021)]{bai2021hu}
白甲丽.
\newblock 呼吸式玻璃幕墙建筑室内光热环境调控方法研究[D].
\newblock 西安: 西安建筑科技大学, 2021.

\bibitem[许钊(2021)]{xu2021xin}
许钊.
\newblock 新型玻璃幕墙光热电能量转化规律研究[D].
\newblock 呼和浩特: 内蒙古工业大学, 2021.

\bibitem[闫好萱~等(2021)]{yan2021yi}
闫好萱, 曹一帆, 冯博泉,等.
\newblock 异形玻璃幕墙清洗机器人设计研究[J].
\newblock 科技创新与应用, 2021, 11(13): 102--104+107.

\bibitem[李雪新~等(2019)]{li2019bo}
李雪新, 靳文停, 葛宜元,等.
\newblock 玻璃幕墙清洗机器人关键技术综述[J].
\newblock 木材加工机械, 2019, 30(6): 14--20.

\bibitem[胡郝君(2019)]{hu2019ji}
胡郝君.
\newblock
  基于涵道风扇的玻璃幕墙清洗机器人设计与姿态控制研究[D].
\newblock 哈尔滨: 哈尔滨工业大学, 2019.

\bibitem[url(2015)]{url2015gao}

\newblock 高楼清洁机器人[EB/OL].
  \url{https://www.zcool.com.cn/work/ZMTMxNDEyODg=.html}, 2015.

\bibitem[mag(2021年12月29日(第003版))]{magazine2021}

\newblock 《“十四五”机器人产业发展规划》解读[N].
  人民邮电, 2021年12月29日(第003版).

\bibitem[Nishi et al.(1986)]{Nishi1986Design}
Nishi A., Wakasugi Y., Watanabe K.
\newblock Design of a robot capable of moving on a vertical wall[J].
\newblock Advanced Robotics, 1986, 1(1): 33--45.

\bibitem[Alkalla et al.(2017)]{alkalla2017tele}
Alkalla M.G., Fanni M.A., Mohamed A.M., {\it{et al}}.
\newblock Tele-operated propeller-type climbing robot for inspection of
  petrochemical vessels[J].
\newblock Industrial Robot: An International Journal, 2017, 44(2): 166--177.

\bibitem[Yoshida et al.(2021)]{yoshida2021hangrawler}
Yoshida T., Yamada Y., Warisawa S., {\it{et al}}.
\newblock HanGrawler 2: Super-high-speed and large-payload ceiling mobile robot
  using crawler[A].
\newblock In: 2021 IEEE/RSJ International Conference on Intelligent Robots and
  Systems (IROS)[C].
\newblock Washington: IEEE, 2021: 2491--2497.

\bibitem[Yang et al.(2016)]{yang2016path}
Yang C.h.J., Paul G., Ward P., {\it{et al}}.
\newblock A path planning approach via task-objective pose selection with
  application to an inchworm-inspired climbing robot[A].
\newblock In: 2016 IEEE International Conference on Advanced Intelligent
  Mechatronics (AIM)[C].
\newblock Washington: IEEE, 2016: 401--406.

\bibitem[Zhao et al.(2018)]{zhao2018obstacle}
Zhao Y., Chai X., Gao F., {\it{et al}}.
\newblock Obstacle avoidance and motion planning scheme for a hexapod robot
  Octopus-III[J].
\newblock Robotics and Autonomous Systems, 2018, 103: 199--212.

\bibitem[于靖军~等(2015)]{yu2015ji}
于靖军, 刘辛军, 丁希仑.
\newblock 机器人机构学的数学基础[M].
\newblock 第2版.
\newblock 北京: 机械工业出版社, 2015.

\bibitem[贠超和王伟(2018)]{yun2018ji}
贠超, 王伟.
\newblock 机器人学导论[M].
\newblock 第4版.
\newblock 北京: 机械工业出版社, 2018.

\bibitem[Maciejowski(2002)]{maciejowski2002predictive}
Maciejowski J.M.
\newblock Predictive control: With constraints[M].
\newblock London: Pearson education, 2002.

\bibitem[Hardy et al.(1988)]{hardy1988inequalities}
Hardy G., Littlewood J., P{\'o}lya G.
\newblock {Inequalities}[M].
\newblock Cambridge Mathematical Library.
\newblock Cambridge: Cambridge University Press, 1988.

\bibitem[贾振中~等(2016)]{jia2016ji}
贾振中, 徐静, 付成龙,等.
\newblock 机器人建模和控制[M].
\newblock 北京: 机械工业出版社, 2016.

\bibitem[Nansai et al.(2018)]{nansai2018design}
Nansai S., Onodera K., Veerajagadheswar P., {\it{et al}}.
\newblock Design and experiment of a novel fa{\c{c}}ade cleaning robot with a
  biped mechanism[J].
\newblock Applied Sciences, 2018, 8(12): 1--17.

\bibitem[Khan et al.(2020)]{khan2020icrawl}
Khan M.B., Chuthong T., Do C.D., {\it{et al}}.
\newblock iCrawl: An inchworm-inspired crawling robot[J].
\newblock IEEE Access, 2020, 8: 200655--200668.

\bibitem[Jiang et al.(2014)]{jiang2014gait}
Jiang Y., Yue Z., Dong W., {\it{et al}}.
\newblock Gait planning of concave transition for a wall-climbing robot[A].
\newblock In: 2014 IEEE International Conference on Information and Automation
  (ICIA)[C].
\newblock Washington: IEEE, 2014: 1284--1288.

\bibitem[付紫杨(2021)]{fu2021shuang}
付紫杨.
\newblock 双足爬壁机器人空间定位与导航[D].
\newblock 广州: 广东工业大学, 2021.

\bibitem[Guan et al.(2016)]{guan2016climbot}
Guan Y., Jiang L., Zhu H., {\it{et al}}.
\newblock Climbot: A bio-inspired modular biped climbing robot—system
  development, climbing gaits, and experiments[J].
\newblock Journal of Mechanisms and Robotics, 2016, 8(2): 1--17.

\bibitem[Zhang et al.(2017)]{zhang2017spatial}
Zhang Y., Su M., Li M., {\it{et al}}.
\newblock A spatial soft module actuated by SMA coil[A].
\newblock In: 2017 IEEE International Conference on Mechatronics and Automation
  (ICMA)[C].
\newblock Washington: IEEE, 2017: 677--682.

\bibitem[Zhu et al.(2020)]{zhu2020planning}
Zhu H., Lu J., Gu S., {\it{et al}}.
\newblock Planning three-dimensional collision-free optimized climbing path for
  biped wall-climbing robots[J].
\newblock IEEE/ASME Transactions on Mechatronics, 2020, 26(5): 2712--2723.

\bibitem[Gu et al.(2018)]{gu2018optimal}
Gu S., Zhu H., Li H., {\it{et al}}.
\newblock Optimal collision-free grip planning for biped climbing robots in
  complex truss environment[J].
\newblock Applied Sciences, 2018, 8(12): 1--22.

\bibitem[苏满佳(2020)]{su2020fang}
苏满佳.
\newblock 仿生软体攀爬机器人的建模、分析与实验[D].
\newblock 广州: 广东工业大学, 2020.

\bibitem[蔡钊雄(2012)]{cai2012ji}
蔡钊雄.
\newblock
  基于多足爬墙机器人平台的桥梁裂缝检测方法研究[D].
\newblock 广州: 华南理工大学, 2012.

\bibitem[谢浩(2015)]{xie2015duo}
谢浩.
\newblock 多足爬墙机器人运动控制及步态规划研究[D].
\newblock 广州: 华南理工大学, 2015.

\bibitem[Ye et al.(2016)]{ye2016degree}
Ye C., Yuan Y., Wei W.
\newblock Degree of freedom analysis of hexapod wall-climbing robot[A].
\newblock In: 6th International Conference on Machinery, Materials,
  Environment, Biotechnology and Computer (MMEBC)[C].
\newblock Paris: Atlantis Press, 2016: 1041--1048.

\bibitem[魏武~等(2016)]{wei2016ji}
魏武, 叶春台, 袁银龙.
\newblock 基于群论的六足机器人运动空间研究[J].
\newblock 机器人, 2016, 38(5): 522--539.

\bibitem[叶春台(2017)]{ye2017ji}
叶春台.
\newblock
  基于螺旋理论与李群的六足机器人运动分析及步态规划[D].
\newblock 广州: 华南理工大学, 2017.

\bibitem[Zhu et al.(2011)]{zhu2011attitude}
Zhu Z., Xia Y., Fu M.
\newblock {Attitude stabilization of rigid spacecraft with finite-time
  convergence}[J].
\newblock International Journal of Robust and Nonlinear Control, 2011, 21(6):
  686--702.


\end{thebibliography}
