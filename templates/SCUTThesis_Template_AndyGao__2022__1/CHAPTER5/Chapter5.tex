%%
%% Copyright: Andy GAO (华工自动化)
%%
%%==========================================================
%%          华南理工大学博士学位论文——第五章
%%       Author: Andy GAO        Date: 2022.01.
%%==========================================================
%%

\chapter{算法-算法-算法-算法-算法-算法-算法-算法-算法-算法-算法}
\label{cha:Chapter5}


\section{引言}
\label{sec:5_Introduction}

引言引言引言引言引言引言引言引言。



\section{算法}
\label{sec:5_EnvironmentModelling}

算法算法算法算法算法,搜索过程如算法{\ref{alg:5_WorkspaceSearching}}所述。
%
%%%%%%%%%%%%%%%%%%%%%%%%%%%%%%%%%%%%%%%%%%%%%%%%%%%%% Algorithm
\begin{algorithm}[!ht]
\Indp               % 增大算法主体的缩进宽度
\SetInd{2em}{0.0em} % sets the size of the space before the vertical rule and after.
\LinesNumbered      % 显示行号
\setstretch{1.5}    % 设置行距为原行距的 1.5 倍
\caption{基于微元法搜索工作空间及其包络算法}
\label{alg:5_WorkspaceSearching}
%%--------------------------------------------------- 算法主体
  \KwIn{各关节转动范围,运动参数上下限(${\beta}_{\rm min}$,${\beta}_{\rm max}$,$pd_{\rm max}$)} % 符号\;输出分号并换行
  \For{吸盘俯仰角 ${\beta} := {\beta}_{\rm min}$ \KwTo ${\beta}_{\rm max}$}{
    \For{极角 ${\gamma} := -{\pi}$ \KwTo ${\pi}$}{
      \For{极径 $pd := 0$ \KwTo $pd_{\rm max}$}{
        {$x_{\rm f} = pd \cdot \cos{\gamma}$,$y_{\rm f} = pd \cdot \sin{\gamma}$;}\\
        {基于逆运动学方程计算各关节变量;}\\
        {检验各关节角度是否满足关节转动约束;}\\
        {记录有效的极径${\Omega}_{\rm p} \leftarrow pd$;}\\
      }
    \If{${\Omega}_{\rm p}$非空}{
      {记录当前有效的极角${\gamma}$;}\\
      {计算并记录有效的$x_{\rm f}$,$y_{\rm f}$,${\beta}$;}\\
      {寻找有效极径的最值;}%\\
      {计算并记录有效的$x_{\rm f}$与$y_{\rm f}$的最值;}\\
    }
    }
  }
  \KwOut{有效工作空间及其包络线}
%%---------------------------------------------------
\end{algorithm}
%%%%%%%%%%%%%%%%%%%%%%%%%%%%%%%%%%%%%%%%%%%%%%%%%%%%% End Algorithm
%


算法算法算法算法算法算法算法算法算法算法。



\section{本章小结}
\label{sec:5_Conclusion}

本章小结本章小结本章小结本章小结本章小结本章小结。



%%
%% Copyright: Andy GAO (华工自动化)
%%
%%==========================================================
%%          华南理工大学博士学位论文——第五章
%%       Author: Andy GAO        Date: 2022.01.
%%==========================================================
%%
